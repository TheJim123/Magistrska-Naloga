\documentclass[mat2]{matdelo}

\avtor{Jimmy Zakeršnik}
\naslov{Omejeni linearni operatorji na Hilbertovih prostorih}
\title{Bounded linear operators on Hilbert spaces}

\mentor{dr.~Daniel~Eremita}

\letnica{2024/25}

%  V povzetku na kratko opišite vsebinske rezultate dela. Sem ne sodi razlaga organizacije dela --
%  v katerem poglavju/razdelku je kaj, pač pa le opis vsebine.
\povzetek{Bo napisan zadnji}

%  Prevod slovenskega povzetka v angleščino.
\abstract{Will be written last}

% navedite vsaj eno klasifikacijsko oznako --
% dostopne so na www.ams.org/mathscinet/msc/msc2020.html
\klasifikacija{47A25, 47B02, 47B15, }
\kljucnebesede{Linearna algebra, funkcionalna analiza, Hilbertov prostor, omejen linearen operator, spekter, spektralni radij, normalen operator, unitaren operator, sebi-adjungiran operator} % navedite nekaj ključnih pojmov, ki nastopajo v delu
\keywords{Linear algebra, funkcional analysis, Hilbert space, bounded linear operator, spectrum, spectral radius, normal operator, unitary operator, self-adjoint operator} % angleški prevod ključnih besed

\zapisiMetaPodatke  % poskrbi za metapodatke in veljaven PDF/A-1b standard

% aktivirajte pakete, ki jih potrebujete
\usepackage{lmodern}
\usepackage{leftidx}
\usepackage{graphicx}
\usepackage{stmaryrd}
\usepackage{wrapfig}
\usepackage{amsthm}
\usepackage{mathrsfs}
\usepackage{silence}
\usepackage{mathtools}
\usepackage{url}
\usepackage{subfigure}
\usepackage{multirow}
\usepackage{lipsum}
\usepackage{wrapfig}
\usepackage{tikz}
\usepackage[format=plain, font=small, labelfont=bf, textfont=it, justification=centerlast]{caption}
\usepackage{booktabs}
\usepackage{siunitx}
\usepackage{enumerate}

% za številske množice uporabite naslednje simbole
\newcommand{\R}{\mathbb{R}}
\newcommand{\F}{\mathbb{F}}
\newcommand{\N}{\mathbb{N}}
\newcommand{\Z}{\mathbb{Z}}
\newcommand{\C}{\mathbb{C}}
\newcommand{\Q}{\mathbb{Q}}
\newcommand{\No}{\N_0}
\newcommand{\n}{\underline{n}}

% matematične operatorje deklarirajte kot take, da jih bo Latex pravilno stavil
% \DeclareMathOperator{\conv}{conv}

% vstavite svoje definicije ...

\newcommand{\abs}[1]{\ensuremath{\lvert #1 \rvert}}
\newcommand{\norm}[1]{\abs{\abs{#1}}}

\newcommand{\Pplus}[1]{\mathbb{#1}_{+}}

\newcommand{\pojem}[1]{\ensuremath{\emph{#1}}}
\newcommand{\con}{\ensuremath{\mathscr{C}}}

\newcommand{\Sp}[2]{\ensuremath{\left<#1, #2\right>}}
\newcommand{\Spp}[2]{\ensuremath{\left<\left<#1, #2\right>\right>}}
\newcommand{\Sppp}[2]{\ensuremath{\left<\left<\left<#1, #2\right>\right>\right>}}


\newcommand{\map}[3]{\ensuremath{{#1}:{#2}\rightarrow{#3}}}
\newcommand{\pra}[3]{{#1}{\ast}({#2}) = {#3}}

\newcommand{\Gen}[1]{\ensuremath{\left<{#1}\right>}}

%===============================================================================
\begin{document}
	%\maketitle je že vključen v class fmfdelo, ki avtomatsko sestavi naslovno stran!
	
	\tableofcontents
	
	\section{Uvod}\label{sect:intro}
	Sem spada uvod, ki bo napisan, ko bo naloga bolj vsebinsko dovršena.
	
	\section{Hilbertovi prostori}
		Tukaj bom povedal osnovno o Hilbertovih prostorih: \begin{itemize}
			\item Izrek o odprti preslikavi (brez dokaza)
			\item Kaj so Hilbertovi prostori
			\item Neenakost $C-S-B$
			\item Ortogonalni komplement
			\item Riezsov izrek
			\item Prostor omejenih linearnih operatorjev nad Hilbertovimi prostori
		\end{itemize}
		
%		Preden se lotimo glavne teme naloge bomo definirali in opisali nekaj osnovnih lastnosti Hilbertovih prostorov. 
%		
%		Spomnimo se najprej definicije vektorskih prostorov in linearne neodvisnosti.
%		
%		\begin{definicija}
%			\label{def:VektSpac}
%			Naj bo $F$ poljubno polje z nevtralnim elementom $0$ in enoto $1$. Neprazna množica $V$, skupaj z operacijama $\map{+}{V\times V}{V}$ in $\map{\cdot}{F\times V}{V}$ je \pojem{vektorski prostor nad $F$}, če velja: \begin{itemize}
%				\item $(V, +)$ je Abelova grupa
%				\item $(\alpha + \beta)\cdot x = \alpha\cdot x + \beta\cdot x;~\forall \alpha, \beta \in F~\&~\forall x\in V$
%				\item $\alpha\cdot(x+y) = \alpha\cdot x + \alpha\cdot y;~\forall \alpha\in F~\&~\forall x, y\in V$
%				\item $\alpha\cdot(\beta\cdot x) = (\alpha\beta)\cdot x;~\forall \alpha, \beta\in F~\&~\forall x\in V$
%				\item $1\cdot x = x;~\forall x\in V$
%			\end{itemize}
%		\end{definicija}
%		
%		\begin{definicija}
%			\label{def:LinNeodv}
%			Naj bo $n\in\N$ in naj bo $V$ poljuben vektorski prostor nad poljubnim poljem $F$. Pravimo, da so vektorji $x_1, x_2, \ldots, x_n \in V$ \pojem{linearno neodvisni}, če enakost $\alpha_1\cdot x_1 + \alpha_2\cdot x_2 + \ldots + \alpha_n\cdot x_n = 0 $ velja le za $\alpha_1 = \alpha_2 = \ldots = \alpha_n = 0$.
%			Če vektorji $x_1, x_2, \ldots, x_n$ niso linearno neodvisni, pravimo, da so \pojem{linearno odvisni}.
%			
%			Naj bo $M$ poljubna neprazna podmnožica vektorskega prostora $V$. Pravimo, da je $M$ \pojem{linearno neodvisna}, če je vsaka njena končna podmnožica linearno neodvisna.
%		\end{definicija}
%		
%		\begin{opomba}
%			\label{op:Polj}
%			V nadaljevanju bomo s $\F$ označili polje, ki je ali polje realnih števil $\R$, ali pa polje kompleksnih števil $\C$. Kadar bo pomembno, bomo natančno navedli, če je $\F = \R$~ali~$\F = \C$.
%		\end{opomba}
%		Sedaj se spomnimo definicij norme in skalarnega produkta, saj bosta v nadaljevanju ta pojma ključna.
%		\begin{definicija}
%			\label{def:Norm}
%			Naj bo $V$ vektorski prostor nad $\F$. Preslikavi $\map{\norm{.}}{V}{\R}$ pravimo \pojem{norma} na $V$, če velja: \begin{itemize}
%				\item $\norm{x} \geq 0;~\forall x\in V$
%				\item $\norm{x} = 0 \iff x = 0$
%				\item $\norm{\alpha\cdot x} = \abs{\alpha}\norm{x};~\forall \alpha\in\F~\&~\forall x\in V$
%				\item $\norm{x+y} \leq \norm{x}+\norm{y};~\forall x, y\in V$
%			\end{itemize}
%			Če je $\norm{.}$ norma na $V$, pravimo, da je $(V, +, \cdot, \norm{.})$ \pojem{normiran prostor} (nad $\F$).
%		\end{definicija}
%		
%		\begin{opomba}
%			\label{op:Norm1}
%			Naj bo $(V, +, \cdot, \norm{.})$ normiran prostor nad poljem $\F$. Enostavno je preveriti naslednje rezultate: \begin{itemize}
%				\item $\norm{0} = 0$
%				\item  $\norm{x_1 + x_2 + \ldots + x_n} \leq \norm{x_1} + \norm{x_2} + \ldots + \norm{x_n};~\forall x_1, x_2, \ldots, x_n\in V$
%				\item $\abs{\norm{x} - \norm{y}} \leq \norm{x-y};~\forall x, y\in V$
%			\end{itemize}
%		\end{opomba}
%		
%		\begin{opomba}
%			\label{op:Norm2}
%			Naj bo $(V, +, \cdot, \norm{.})$ normiran prostor nad poljem $\F$. Enostavno je preveriti, da je s predpisom $d(x, y) = \norm{x - y} ~\forall x, y \in V$ definirana metrika na $V$. Za to metriko pravimo, da je \pojem{porojena z normo} $\norm{.}$.
%		\end{opomba}
%
%		Dejstvo, da vsaka norma porodi metriko nas motivira, da obravnavamo konvergenco tudi v normiranih prostorih.
%		\begin{definicija}
%			\label{def:NormKonv}
%			Naj bo $(V, \norm{.})$ normiran prostor nad poljem $\F$. Naj bo $\bar{x} = \left(x_n\right)_{n\in\N}$ zaporedje s členi iz $V$.
%			\begin{itemize}
%				\item Pravimo, da je zaporedje $\bar{x}$ \pojem{konvergentno}, če obstaja tak $x\in V$, da za vsak $\varepsilon > 0$ obstaja tak $n_0\in \N$, da za vsak $n\in \N$ velja: $n\geq n_0 \Rightarrow \norm{x_n - x} < \varepsilon$. V tem primeru pravimo, da je $x$ \pojem{limita zaporedja} $\left(x_n\right)_{n\in\N}$ in pišemo $\lim_{n\to\infty}x_n = x$.
%				\item Pravimo, da je zaporedje $\bar{x}$ \pojem{Cauchyjevo}, če za vsak $\varepsilon >0$ obstaja tak $n_0\in\N$, da za poljubna $m, n \in \N$ velja: $m, n \geq n_0 \Rightarrow \norm{x_m - x_n} < \varepsilon$.
%				\item Naj bo $\bar{s}$ zaporedje podano s predpisom $s_n = \sum_{k = 1}^{n}x_k;~\forall n\in\N$. Pravimo, da je vrsta $\sum_{k = 1}^{\infty}x_k$ \pojem{konvergentna}, če je konvergentno zaporedje $\bar{s}$. Če je $s$ limita zaporedja $\bar{s}$, tedaj pravimo, da je $s$ \pojem{vsota vrste} $\sum_{k = 1}^{\infty}x_k$ in pišemo $s = \sum_{k = 1}^{\infty}x_k$.
%				\item Pravimo, da je vrsta $\sum_{k = 1}^{\infty}x_k$ \pojem{absolutno konvergentna}, če je vrsta $\sum_{k = 1}^{\infty}\norm{x_k}$ konvergentna.
%			\end{itemize} 
%		\end{definicija}
%		Naslednja trditev nam pove, da za limite v normiranih prostorih veljajo analogi nekaterih rezultatov, ki so nam znani že iz obravnave realnih zaporedij.
%		\begin{trditev}
%			\label{trd:Normlim}
%			Naj bo $(V, \norm{.})$ normiran prostor nad $\F$. Naj bosta $\left(x_n\right)_{n\in\N}$ in $\left(y_n\right)_{n\in\N}$ poljubni konvergentni zaporedji s členi iz $V$ in z limitama $x$ ter $y$. Naj bo $\left(\alpha_n\right)_{n\in\N}$ poljubno konvergentno zaporedje s členi iz $\F$ z limito $\alpha$. Tedaj velja: \begin{enumerate}[i)]
%				\item $\lim_{n\to\infty}(x_n + y_n) = x + y$
%				\item $\lim_{n\to\infty}\norm{x_n} = \norm{\lim_{n\to\infty}x_n}=\norm{x}$
%				\item $\lim_{n\to\infty}(\alpha_n\cdot x_n) = \alpha\cdot x$
%			\end{enumerate}
%		\end{trditev}
%		
%		\begin{dokaz}
%			\begin{enumerate}[i)]
%				\item Naj bo $\varepsilon > 0$ poljuben. Ker sta zaporedji $\left(x_n\right)_{n\in\N}$ in $\left(y_n\right)_{n\in\N}$ konvergentni z limitama $x$ in $y$, obstajata taka $n_1, n_2\in\N$, da vse $n\in\N$, ki so večji ali enaki $n_1$, velja $\norm{x_n - x} < \frac{\varepsilon}{2}$ in za vse $n\in\N$, ki so večji ali enaki $n_2$ velja $\norm{y_n - y} < \frac{\varepsilon}{2}$. Naj bo $n_0 = \max\{n_1, n_2\}$. Tedaj je $\norm{x_n + y_n - (x + y)} = \norm{(x_n - x) + (y_n - y)} \leq \norm{x_n - x} + \norm{y_n - y} < \frac{\varepsilon}{2} + \frac{\varepsilon}{2} = \varepsilon$ za vse $n\in\N$, ki so večji ali enaki $n_0$. Sledi, da je zaporedje $\left(x_n + y_n\right)_{n\in\N}$ konvergentno z limito $x + y$.
%				\item Najprej opazimo, da je zaporedje $\left(\norm{x_n - x}\right)$ realno konvergentno zaporedje z limito $0$, saj je zaporedje $\left(x_n\right)$ konvergentno z limito $x$. Po tretji točki iz opombe \ref{op:Norm1} za vsak $n\in\N$ velja $0 \leq \abs{\norm{x_n} - \norm{x}} \leq \norm{x_n - x}$. Po pravilu o sendviču zato sklepamo, da je $\lim_{n\to\infty}\abs{\norm{x_n} - \norm{x}} = 0$. Po znanem rezultatu iz Analize $1$ potem velja, da je tudi $\lim_{n\to\infty}(\norm{x_n} - \norm{x}) = 0$. Sledi, da je $\lim_{n\to\infty}\norm{x_n} - \norm{x} = 0$ oziroma $\lim_{n\to\infty}\norm{x_n} = \norm{x}$.
%				\item Ponovno upoštevamo, da je $\norm{x_n - x}$ konvergentno realno zaporedje z limito $0$, ter na enak način vidimo, da je $\abs{\alpha_n - \alpha}$ konvergentno realno zaporedje z limito $0$. Dodatno vidimo, da za vsak $n\in\N$ velja: $$0 \leq \norm{\alpha_n\cdot x_n - \alpha\cdot x} \leq \norm{\alpha_n\cdot x_n - \alpha\cdot x_n + \alpha\cdot x_n - \alpha\cdot x}$$
%				Skrajno desno normo sedaj po trikotniškem pravilu ocenimo navzgor z $$\norm{\alpha_n\cdot x_n - \alpha\cdot x_n} + \norm{\alpha\cdot x_n - \alpha\cdot x}=\abs{\alpha_n - \alpha}\norm{x_n} + \abs{\alpha}\norm{x_n - x}$$ Velja torej ocena: \begin{equation}
%					\label{eq:Normlim3}
%					0 \leq \norm{\alpha_n\cdot x_n - \alpha\cdot x} \leq \abs{\alpha_n - \alpha}\norm{x_n} + \abs{\alpha}\norm{x_n - x}
%				\end{equation}
%				Sedaj opazimo, upoštevajoč prejšnjo točko, da je $\abs{\alpha_n - \alpha}\norm{x_n}$ produkt dveh konvergentnih realnih zaporedij in velja: $$\lim_{n\to\infty}\abs{\alpha_n - \alpha}\norm{x_n} = \lim_{n\to\infty}\abs{\alpha_n - \alpha} \cdot \lim_{n\to\infty}\norm{x_n} = 0\cdot\norm{x} = 0$$
%				Dodatno vidimo, da je $\lim_{n\to\infty}\abs{\alpha}\norm{x_n - x} = \abs{\alpha}\lim_{n\to\infty}\norm{x_n - x} = \abs{\alpha}\cdot 0 = 0$. Potem sledi, da je zaporedje $\left(\abs{\alpha_n - \alpha}\norm{x_n} + \abs{\alpha}\norm{x_n - x}\right)_{n\in\N}$ realno konvergentno zaporedje z limito $0$. Po pravilu o sendviču, upoštevajoč oceno \eqref{eq:Normlim3}, sklepamo, da je potem $\lim_{n\to\infty}\norm{\alpha_n\cdot x_n - \alpha\cdot x} = 0$, od tod pa sledi, da za vsak $\varepsilon > 0$ obstaja $n_0\in\N$, da za vsak $n\in\N$, ki je večji ali enak $n_0$, velja $\norm{(\alpha_n\cdot x_n - \alpha\cdot x) - 0}=\norm{\alpha_n\cdot x_n - \alpha\cdot x} < \varepsilon$. Po definiciji konvergence zaporedja potem sledi, da je $\lim_{n\to\infty}(\alpha_n\cdot x_n) = \alpha\cdot x$.
%			\end{enumerate}
%		\end{dokaz}
		\begin{lema}
			\label{lem:invnorm}
			Za poljubna normirana prostora $X$ ter $Y$ in poljuben $T\in B(X, Y)$ velja: Če je $T$ obrnljiv, je $\norm{Tx}\geq \norm{T^{-1}}^{-1}\norm{x};~\forall x\in X$.
		\end{lema}
		
		\begin{lema}
			\label{lem:banzap}
			Za poljuben Banachov prostor $X$, poljuben normiran prostor $Y$ ter poljuben $T\in B(X, Y)$ velja: Če obstaja tak $r>0$, da je $\norm{Tx}\geq r\norm{x};~\forall x\in X$, je $ImT$ zaprti podprostor v $Y$.
		\end{lema}
		
		\begin{definicija}
			\label{def:ScalProd}
			Naj bo $V$ vektorski prostor nad $\F$. Preslikavi $\map{\Sp{.}{.}}{V\times V}{\R}$ pravimo \pojem{skalarni produkt} na $V$, če velja: \begin{itemize}
				\item $\Sp{x}{x} \geq 0;~\forall x\in V$
				\item $\Sp{x}{x} = 0 \iff x = 0$
				\item $\Sp{x}{y} = \overline{\Sp{y}{x}};~\forall x, y\in V$
				\item $\Sp{\alpha_1\cdot x_1 + \alpha_2\cdot x_2}{y} = \alpha_1\Sp{x_1}{y} +\alpha_2\Sp{x_2}{y};~\forall \alpha_1, \alpha_2\in\F~\&~\forall x_1, x_2, y\in V$
			\end{itemize}
			Če je $V$ vektorski prostor (nad $\F$) in $\Sp{.}{.}$ skalarni produkt na njem, pravimo, da je $(V, +, \cdot, \Sp{.}{.})$ \pojem{prostor s skalarnim produktom}.
		\end{definicija}
		
		\begin{opomba}
			\label{op:ScalProd}
			Naj bo $(V, +, \cdot, \Sp{.}{.})$ prostor s skalarnim produktom nad poljem $\F$. Enostavno je preveriti, da je s predpisom $\norm{x} = \sqrt{\Sp{x}{x}}~\forall x\in V$ definirana norma na $V$. Za tako normo pravimo, da je \pojem{porojena s skalarnim produktom} $\Sp{.}{.}$.
		\end{opomba}
		\begin{definicija}
			\label{def:PolnMP}
			Naj bo $(M, d)$ poljuben metrični prostor. Pravimo, da je $M$ \pojem{poln} metrični prostor, če je vsako Cauchyjevo zaporedje v $M$ konvergentno v $M$.
		\end{definicija}
		Sedaj definirajmo Hilbertove prostore.
		\begin{definicija}
			\label{def:Hilbert}
			Naj bo $(V, \Sp{.}{.})$ poljuben prostor s skalarnim produktom nad poljem $\F$. Naj bo $\norm{.}$ norma na $V$ porojena s $\Sp{.}{.}$ in naj bo $\map{d}{V\times V}{\R}$ metrika na $V$ porojena z $\norm{.}$. Če je $(V, d)$ poln metrični prostor, pravimo, da je $(V, \Sp{.}{.})$ \pojem{Hilbertov prostor} nad $\F$. 
		\end{definicija}
		
		\begin{posledica}
			\label{posl:zaportoggost}
			Za poljuben zaprti podprostor $Y$ poljubnega Hilbertovega prostora velja $Y^{\bot\bot} = Y$.
		\end{posledica}
		
		\begin{posledica}
			\label{posl:ortoggost}
			Za poljuben podprostor $Y$ poljubnega Hilbertovega prostora velja $Y^{\bot\bot} = \overline{Y}$.
		\end{posledica}
		
	\section{Omejeni linearni operatorji nad Hilbertovimi prostori}
		V tem poglavju bomo obravnavali lastnosti omejenih linearnih operatorjev nad Hilbertovimi prostori ter lastnosti njihovih spektrov, kadar sta domena in kodomena kompleksna Hilbertova prostora. Posebej nas bo zanimala operacija adjungiranja, zato bomo najprej povedali nekaj o njej.
		\subsection{Adjungirani operatorji}
			\begin{izrek}
				\label{izr:Adjexist}
				Naj bosta $(X, \Spp{.}{.})$ in $(Y, \Sp{.}{.})$ poljubna kompleksna Hilbertova prostora in naj bo $T$ poljuben element $B(X, Y)$. Tedaj obstaja enolično določen operator $T^*\in B(Y, X)$, za katerega velja $$\Sp{Tx}{y} = \Spp{x}{T^*y};~\forall x\in X~\&~\forall y \in Y $$
			\end{izrek}
			\begin{dokaz}
				Naj bo $y\in Y$ poljuben ter naj bo $\map{f_y}{X}{\C}$ preslikava definirana s predpisom $f_y(x) = \Sp{Tx}{y};~\forall x\in X$. Opazimo, da je $f_y$ linearen funkcional nad $X$. Po neenakosti Cauchy-Schwarz-Bunyakowsky potem velja: $$\abs{f_y(x)} = \abs{\Sp{Tx}{y}} \leq \norm{Tx}_Y \cdot \norm{y}_Y \leq \norm{T}\cdot\norm{x}_X\cdot\norm{y}_Y;~ \forall x\in X$$ To pomeni, da je $f_y$ omejen linearen funkcional nad $X$. Posledično velja, po Riezsovem izreku, da obstaja enolično določen $z\in X$, tak, da je $f_y(x) = \Spp{x}{z};~\forall x\in X$. Sedaj definiramo preslikavo $\map{T^*}{Y}{X}$, ki vsakemu $y\in Y$ priredi pripadajoči $z\in X$ (tisti, enolično določen, za katerega je $f_y(x) = \Spp{x}{z};~\forall x\in X$). Posledično velja: $$\Sp{Tx}{y}=\Spp{x}{T^*y}~\forall x\in X,~\forall y\in Y$$
				Sedaj preverimo, da je $T^*$ omejen linearen operator. Naj bodo $y_1, y_2 \in Y$ in $x\in X$ poljubni elementi ter $\alpha, \beta \in \C$ poljubna skalarja. Tedaj je: \begin{align*}
					\Spp{x}{T^*(\alpha y_1 + \beta y_2)} &= \Sp{Tx}{\alpha y_1 + \beta y_2} \\
					&= \bar{\alpha}\Sp{Tx}{y_1} + \bar{\beta}\Sp{Tx}{y_2} \\
					&= \bar{\alpha}\Spp{x}{T^*y_1} + \bar{\beta}\Spp{x}{T^*y_2} \\
					&= \Spp{x}{\alpha T^*y_1} + \Spp{x}{\beta T^*y_2}
				\end{align*}
				Sledi, da je $T^*(\alpha y_1 + \beta y_2)=\alpha T^*y_1 + \beta T^*y_2;~\forall x\in X, \forall y_1, y_2 \in Y, \forall \alpha, \beta \in \C$, torej je $T^*$ res linearna preslikava. Dodatno, velja: $$\norm{T^*y}_X^2 = \Spp{T^*y}{T^*y} = \Sp{TT^*y}{y} \leq \norm{TT^*y}_Y\norm{y}_Y \leq \norm{T}\norm{T^*y}_X\norm{y}_Y;~\forall y\in Y$$
				Če je $\norm{T^*y}_X \neq 0$, dobljeno neenakost delimo z $\norm{T^*y}_X$ ter tako dobimo oceno $\norm{T^*y}_X \leq \norm{T}\norm{y}_Y$. V primeru, ko je $\norm{T^*y}_X = 0$, prejšnja ocena velja trivialno. Sledi, da je $$\norm{T^*y}_X \leq \norm{T}\norm{y}_Y; \forall y\in Y$$
				Preslikava $T^*$ je torej omejen linearen operator in velja $\norm{T^*}\leq \norm{T}$. Za konec še pokažimo, da je $T^*$ enolično določen. Denimo, da imamo $U_1, U_2 \in B(Y, X)$, za katera velja: $\Sp{Tx}{y} = \Spp{x}{U_1 y} = \Spp{x}{U_2 y};~\forall x\in X, \forall y\in Y$. Tedaj velja, da je $U_1y = U_2y;~\forall y\in Y$, torej je $U_1 = U_2$. Sledi, da je $T^*$ res enolično določen.
			\end{dokaz}
			Izrek \ref{izr:Adjexist} nas motivira, da vpeljemo naslednjo definicijo.
			\begin{definicija}
				\label{def:adjop}
				Naj bosta $X$ in $Y$ kompleksna Hilbertova prostora ter naj bo $T\in B(X, Y)$. Tedaj operatorju $T^*$ iz izreka \ref{izr:Adjexist} pravimo \pojem{adjungiran operator} operatorja $T$.
			\end{definicija}
			Sedaj bomo navedli in dokazali nekatere pomembne lastnosti adjungiranja.
			
			\begin{izrek}
				\label{izr:adjlast}
				Naj bodo $(X, \Spp{.}{.})$, $(Y, \Sp{.}{.})$ in $(Z, \Sppp{.}{.})$ poljubni kompleksni Hilbertovi prostori ter naj bodo $U, V \in B(X, Y)$ ter $T \in B(Y, Z)$ poljubni omejeni linearni operatorji. Tedaj velja: \begin{enumerate}[a)]
					\item $(\alpha U + \beta V)^* = \bar{\alpha}U^* + \bar{\beta}V^*; \forall \alpha, \beta \in \C$
					\item $(TU)^* = U^*T^*$
					\item $(U^*)^* = U$
					\item $\norm{U^*} = \norm{U}$
					\item Preslikava $\map{f}{B(X, Y)}{B(Y, X)}$, definirana s predpisom $f(U) = U^*$ je zvezna.
					\item $\norm{U^*U} = \norm{U}^2$
				\end{enumerate}
			\end{izrek}
			
			\begin{dokaz}
				\begin{enumerate}[a)]
					\item Naj bodo $U, V \in B(X, Y)$ in $\alpha, \beta \in \C$ poljubni. Tedaj $\forall x\in X, \forall y\in Y$ velja: \begin{align*}
						\Spp{x}{(\alpha U + \beta V)^*y} &= \Sp{(\alpha U + \beta V)x}{y} \\
						&= \alpha \Sp{Ux}{y} + \beta \Sp{Vx}{y} \\
						&= \alpha \Spp{x}{U^*y} + \beta\Spp{x}{V^*y} \\
						&= \Spp{x}{\bar{\alpha}U^*y} + \Spp{x}{\bar{\beta}V^*y} \\
						&= \Spp{x}{(\bar{\alpha} U^* + \bar{\beta} V^*)y}
					\end{align*} 
					Sledi, da je $(\alpha U + \beta V)^* = (\bar{\alpha} U^* + \bar{\beta} V^*)$.
					\item Naj bosta $U\in B(X, Y)$ in $T\in B(Y, Z)$ poljubna. Tedaj $\forall x\in X, \forall z\in Z$ velja: $$\Spp{x}{(TU)^*z} = \Sppp{(TU)x}{z} = \Sppp{T(Ux)}{z} = \Sp{Ux}{T^*z} = \Spp{x}{U^*T^*z}$$
					Posledično je res $(TU)^* = U^*T^*$.
					\item Naj bo $U\in B(X, Y)$ poljuben. Tedaj $\forall x\in X, \forall y\in Y$ velja: $$\Sp{y}{(U^*)^*x} = \Spp{U^*y}{x} = \overline{\Spp{x}{U^*y}} = \overline{\Sp{Ux}{y}} = \Sp{y}{Ux}$$
					Posledično je $(U^*)^* = U$.
					\item V dokazu izreka \ref{izr:Adjexist} smo že pokazali, da je $\norm{U^*} \leq \norm{U}$. Ko upoštevamo prejšnjo točko, vidimo, da velja: $$\norm{U} = \norm{(U^*)^*} \leq \norm{U^*} \leq \norm{U}$$
					Posledično sledi, da je $\norm{U^*} = \norm{U}$.
					\item Naj bodo $U, V \in B(X, Y)$ $\varepsilon > 0$ poljubni in izberemo $\delta = \varepsilon$. Denimo, da je $\norm{U - V} < \delta$. Tedaj je $\norm{f(U) - f(V)} = \norm{U^* - V^*} = \norm{(U - V)^*}$. Po prejšnji točki je $\norm{(U-V)^*} = \norm{U - V} < \delta = \varepsilon$, torej je $f$ zvezna.
					\item Naj bo $U\in B(X, Y)$ poljuben. Ker že vemo, da je $\norm{U^*} = \norm{U}$, hitro vidimo, da je $\norm{U^*U}\leq\norm{U^*}\norm{U} = \norm{U}^2$.
					Hkrati vidimo, da za $\forall x\in X$ velja $\norm{Ux}_Y^2 = \Sp{Ux}{Ux} = \Spp{U^*Ux}{x}$. Ko upoštevamo neenakost Cauchy-Schwarz-Bunyakowsky, dobimo oceno $\Spp{U^*Ux}{x} \leq \norm{U^*Ux}_X\norm{x}_X \leq \norm{U^*U}\norm{x}_X^2$. Sledi ocena: $\norm{Ux}_Y^2 \leq \norm{U^*U}\norm{x}_X^2 \forall x\in X$, torej je $\norm{U}^2 \leq \norm{U^* U}$. Posledično velja iskana enakost.
				\end{enumerate}
			\end{dokaz}
			Pri obravnavi adjungiranih operatorjev nam bodo prav prišli naslednji pomožni rezultati.
			\begin{lema}
				\label{lem:adjlast2}
				Naj bosta $(X, \Spp{.}{.})$ in $(Y, \Sp{.}{.})$ poljubna kompleksna Hilbertova prostora ter naj bo $T\in B(X, Y)$ poljuben. Tedaj velja: \begin{enumerate}[a)]
					\item $KerT = (ImT^*)^{\bot}$
					\item $KerT^* = (ImT)^{\bot}$
					\item $KerT^* = \{0\} \iff ImT$ je gosta v $Y$
				\end{enumerate}
			\end{lema}
			
			\begin{dokaz}
				\begin{enumerate}[a)]
					\item Najprej pokažimo, da je $KerT \subseteq (ImT^*)^\bot$. Naj bo $x\in KerT$ poljuben ter izberemo poljuben $z\in ImT^*$. Potem za $z$ obstaja $y\in Y$, da je $T^*y = z$. Posledično, je $$\Spp{x}{z} = \Spp{x}{T^*y} = \Sp{Tx}{y} = \Sp{0}{y} = 0$$
					Sledi, da je $x\in (ImT^*)^\bot$. Ker je $x$ bil poljuben, premislek velja za vsak $x\in KerT$, torej je $KerT \subseteq (ImT^*)^\bot$.
					Naj bo sedaj $x\in (ImT^*)^\bot$ poljuben. Ker je $T^*Tx \in ImT^*$, velja: $$\norm{Tx}^2 = \Sp{Tx}{Tx} = \Sp{x}{T^*Tx}= 0$$
					Sledi, da je $Tx = 0$ oz. $x\in KerT$. Ker je bil $x$ poljuben, sledi $(ImT^*)^\bot \subseteq KerT$. Posledično velja enakost.
					\item Da dokažemo ta rezultat, upoštevamo prejšnjo točko tega dokaza za $T^*$ ter točko $c)$ izreka \ref{izr:adjlast}. Posledično je $KerT^* = (Im(T^*)^*)^\bot = (ImT)^\bot$.
					\item Da dokažemo ta rezultat bomo dokazali, da veljata implikaciji v obe smeri. \begin{itemize}
						\item[$\Rightarrow)$] Denimo, da je $KerT^* = \{0\}$. Potem vidimo, upoštevajoč posledico \ref{posl:ortoggost} ter točko $a)$ tega dokaza, da velja: $$\overline{ImT} = ((ImT)^\bot)^\bot = (KerT^*)^\bot = \{0\}^\bot = Y$$
						Sledi, da je $ImT$ gosta v $Y$ po definiciji.
						\item[$\Leftarrow)$] Denimo, da je $ImT$ gosta v $Y$. Potem po posledici \ref{posl:ortoggost} velja, da je $$((ImT)^{\bot})^\bot = \overline{ImT} = Y$$ Po točki $b)$ tega dokaza velja $KerT^* = (ImT)^\bot$. Ker je $ImT\subseteq Y$ neprazna podmnožica, velja, da je $(ImT)^\bot$ zaprti podprostor v $Y$. Po posledici \ref{posl:zaportoggost} potem sledi: $$KerT^* = (ImT)^\bot =(((ImT)^\bot)^\bot)^\bot = Y^\bot = \{0\}$$
					\end{itemize} 
					
				\end{enumerate}
			\end{dokaz}

			\begin{posledica}
				\label{posl:adjinv}
				Naj bo $X$ poljuben kompleksen Hilbertov prostor ter naj bo $T\in B(X)$. Naslednji trditvi sta ekvivalentni: \begin{enumerate}
					\item Operator $T$ je obrnljiv.
					\item $KerT^* = \{0\}$ ter obstaja tak $r > 0$, da je $\norm{Tx} \geq r\norm{x}~\forall x\in X$.
				\end{enumerate}
			\end{posledica}
			
			\begin{dokaz}
				Dokazali bomo obe implikaciji.
				\begin{itemize}
					\item[$(1)\Rightarrow (2)$] Denimo, da je $T$ obrnljiv. Tedaj je $T$ bijekcija in njegov inverz, $T^{-1}$ pripada $B(X)$. Posledično je $ImT = Y$. Ker je $X$ Hilbertov prostor, je tudi normiran prostor, torej po lemi \ref{lem:invnorm} velja, da je $\norm{Tx}\geq \norm{T^{-1}}^{-1}\norm{x};~\forall x\in X$. Določimo $r = \norm{T^{-1}}^{-1}$. Ker je $X$ tudi Banachov prostor, po lemi \ref{lem:banzap} velja, da je $ImT$ zaprti podprostor v $Y$. Sledi, da je $ImT = \overline{ImT} = Y$, torej je $ImT$ gost v $Y$. Po lemi \ref{lem:adjlast2} je potem $KerT^* = \{0\}$.
					\item[$(1)\Leftarrow (2)$] Denimo sedaj, da je $KerT^* = \{0\}$ in da obstaja tak $r>0$, da za $\forall x\in X$ velja $\norm{Tx} \geq r\norm{x}$. Po lemi \ref{lem:adjlast2} potem sledi, da je $ImT$ gost v $Y$. Lema \ref{lem:banzap} nam tudi pove, da je $ImT$ zaprti podprostor v $Y$. Sledi, da je $ImT = \overline{ImT} = Y$.
					Naj bo sedaj $x\in KerT$ poljuben. Tedaj je $Tx = 0$, po predpostavki pa potem sledi, da je $0 = \norm{Tx} \geq r\norm{x}$. To je pa možno le, kadar je $x = 0$. Sledi, da je $KerT = \{0\}$, torej je $T$ bijektiven endomorfizem nad Banachovim prostorom $X$. Po izreku o odprti preslikavi sledi, da je $T$ obrnljiv.
				\end{itemize}
			\end{dokaz}
			Za konec tega podpoglavja premislimo, v sledeči lemi, kako sta povezani operaciji invertiranja in adjungiranja.
			\begin{lema}
				\label{lem:adjinv}
				Naj bo $X$ poljuben kompleksen Hilbertov prostor ter naj bo $T\in B(X)$ poljuben. Če je $T$ obrnljiv, je obrnljiv tudi $T^*$.
			\end{lema}
		
			\begin{dokaz}
				Naj bo $T\in B(X)$ poljuben obrnljiv operator. Tedaj velja, da je $TT^{-1} = T^{-1}T = I$. Enačbo adjungiramo ter tako dobimo $(TT^{-1})^* = (T^{-1}T)^* = I^*$. Ko upoštevamo točko $b)$ izreka \ref{izr:adjlast}, se enačba poenostavi v: $$(T^{-1})^*T^* = T^*(T^{-1})^* = I$$
				Posledično je $T^*$ obrnljiv ter $(T^*)^{-1} = (T^{-1})^*$.
			\end{dokaz}
		\subsection{Normalni operatorji}
			Sedaj se bomo posvetili, glede na adjungiranje, posebnim primerom operatorjev. Prvi izmed teh, so t.~i.~ normalni operatorji.
			\begin{definicija}
				\label{def:normalop}
				Naj bo $X$ poljuben kompleksen Hilbertov prostor in $T$ poljuben omejen linearen operator nad $X$. Pravimo, da je $T$ \pojem{normalen}, če je $TT^* = T^*T$.
			\end{definicija}
			
			\begin{lema}
				\label{lem:normalnorm}
				Naj bo $(X, \Sp{.}{.})$ poljuben kompleksen Hilbertov prostor. Naj bo $T\in B(X)$ poljuben normalen operator ter $r > 0$ poljuben. Velja: \begin{enumerate}[a)]
					\item $\norm{Tx} = \norm{T^*x};~\forall x\in X$.
					\item Če je $\norm{Tx} \geq r\norm{x};~\forall x\in X$, je $KerT^* = \{0\}$.
				\end{enumerate}
			\end{lema}
			\begin{dokaz}
				\begin{enumerate}[a)]
					\item Naj bo $x\in X$ poljuben. Potem je \begin{align*}
						\norm{Tx}^2 - \norm{T^*x}^2 &= \Sp{Tx}{Tx} - \Sp{T^*x}{T^*x} = \Sp{T^*Tx}{x} - \Sp{TT^*x}{x}\\
						&= \Sp{T^*Tx - TT^*x}{x} = \Sp{(T^*T - TT^*)x}{x}
					\end{align*}
					Ker je $T$ normalen, je $TT^* = T^*T$, torej je $$\norm{Tx}^2 - \norm{T^*x}^2 = \Sp{(T^*T - TT^*)x}{x} = \Sp{0}{x} = 0 $$
					Posledično je $\norm{Tx} - \norm{T^*x}$. Ker je bil $x$ poljuben, enakost velja za $\forall x\in X$.
					\item Denimo, da obstaja tak $r>0$, da je $\norm{Tx} \geq r\norm{x};~\forall x\in X$. Naj bo $x\in KerT^*$ poljuben. Po točki $a)$ tega dokaza je potem $0 = \norm{T^*x} = \norm{Tx} \geq r\norm{x}$. Sledi, da je $\norm{x} = 0$ oz. $x = 0$. Posledično je $KerT^* = \{0\}$.
				\end{enumerate}
			\end{dokaz}
			Na podlagi posledice \ref{posl:adjinv} in leme \ref{lem:normalnorm} sledi naslednja posledica.
			\begin{posledica}
				\label{posl:normalinv}
				Naj bo $X$ poljuben kompleksen Hilbertov prostor in naj bo $T\in B(X)$ poljuben normalen operator. Naslednji trditvi sta ekvivalentni: \begin{enumerate}
					\item Operator $T$ je obrnljiv.
					\item Obstaja tak $r > 0$, da je $\norm{Tx} \geq r\norm{x}~\forall x\in X$.
				\end{enumerate}
			\end{posledica}
			\begin{dokaz}
				Dokazali bomo implikaciji v obe smeri.
				\begin{itemize}
					\item[$(1)\Rightarrow (2)$] Velja direktno po posledici \ref{posl:adjinv}.
					\item[$(1)\Leftarrow (2)$] Denimo, da obstaja tak $r>0$, da je $\norm{Tx} \geq r\norm{x}~\forall x\in X$. Po lemi \ref{lem:normalnorm} je potem $KerT^* = \{0\}$. Po posledici \ref{posl:adjinv} je potem $T$ obrnljiv.
				\end{itemize}
			\end{dokaz}
			\subsection{Sebi-adjungirani operatorji}
			Pri obravnavi preslikav nas pogosto zanimajo t.~i.~ ">fiksne točke"< - točke, ki jih preslikava preslika nazaj vase. V primeru adjungiranja, operatorjem, ki jih operacija ">fiksira"< damo posebno ime.
			\begin{definicija}
				\label{def:sebiadjop}
				Naj bo $X$ poljuben kompleksen Hilbertov prostor in naj bo $T\in B(X)$ poljuben. Pravimo, da je $T$ \pojem{sebi-adjungiran}, če velja $T = T^*$.
			\end{definicija}
			V naslednji lemi so povzete nekatere osnovne lastnosti sebi-adjungiranih operatorjev.
			\begin{lema}
				\label{lem:sebiadjlast}
				Naj bo $X$ poljuben kompleksen Hilbertov prostor. Naj bo $\mathcal{S}(X)$ množica vseh sebi-adjungiranih operatorjev znotraj $B(X)$ in naj bo $T\in B(X)$ poljuben. Velja:
				\begin{enumerate}[a)]
					\item Za poljubni $\alpha, \beta \in \R$ in poljubna $T_1, T_2 \in \mathcal{S}(X)$ velja: $\alpha T_1 + \beta T_2 \in \mathcal{S}(X)$
					\item Množica $\mathcal{S}(X)$ je zaprta podmnožica $B(X)$.
					\item Operatorja $TT^*$ in $T^*T$ sta sebi-adjungirana.
					\item $T = R + iS$, za neka sebi-adjungirana operatorja $R, S \in \mathcal{S}(X)$.
				\end{enumerate}
			\end{lema}
			\begin{dokaz}
				\begin{enumerate}[a)]
					\item Naj bosta $T_1, T_2 \in \mathcal{S}(X)$ poljubna. Ker sta operatorja sebi-adjungirana, po prvi točki izreka \ref{izr:adjlast} velja: $(\alpha T_1 + \beta T_2)^* = \alpha T_1^* + \beta T_2^* = \alpha T_1 + \beta T_2$. Posledično je $\alpha T_1 + \beta T_2 \in \mathcal{S}(X)$.
					\item Naj bo $\{T_n\}_{n\in\N}$ konvergentno zaporedje s členi v $\mathcal{S}(X)$ in limito $T\in B(X)$. Predzadnja točka izreka \ref{izr:adjlast} nam pove, da je adjungiranje zvezno, torej je tudi zaporedje $\{T_n^*\}_{n\in\N}$ konvergentno z limito $T^* \in B(X)$. Ker so $T_n \in \mathcal{S}(X)~ \forall n\in \N$, velja $T_n^* = T_n~\forall n\in \N$, torej je $T^* = T$. Posledično je tudi $T\in \mathcal{S}(X)$. Sledi, da je $\mathcal{S}(X)$ zaprta.
					\item Naj bo $T\in B(X)$ poljuben. Tedaj je, upoštevajoč točko $c)$ izreka \ref{izr:adjlast}, $(T^*T)^* = T^*T^{**} = T^*T$. Sledi, da je $T^*T\in \mathcal{S}(X)$ in na enak način premislimo, da enako velja za $TT^*$.
					\item Naj bo $T\in B(X)$ poljuben. Definiramo operatorja $R$ in $S$ na naslednji način: $$R = \frac{1}{2}(T + T^*),~ S=\frac{1}{2i}(T - T^*)$$
					Očitno velja, da je $T = R + iS$. Preverimo še, da sta $R$ in $S$ res sebi-adjungirana.
					\begin{align*}
						R^* &= (\frac{1}{2}(T + T^*))^* = \frac{1}{2}(T^* + T^{**}) = \frac{1}{2}(T^* + T) = R \\
						S^* &= (\frac{1}{2i}(T - T^*))^* = \frac{-1}{2i}(T - T^*)^* = \frac{-1}{2i}(T^* - T^{**}) =\frac{1}{2i}(T - T^*) = S
					\end{align*}
				\end{enumerate}
			\end{dokaz}
	\begin{thebibliography}{99}
		\bibitem{bib:Rynne} B.~P.~Rynne,~M.~A.~Youngson, \emph{Linear functional analysis},~Springer-Verlag London,~Ltd.,~London,~2008.
	\end{thebibliography}
\end{document}