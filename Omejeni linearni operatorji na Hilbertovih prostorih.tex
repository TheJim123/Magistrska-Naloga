\documentclass[mat2]{matdelo}

\avtor{Jimmy Zakeršnik}
\naslov{Omejeni linearni operatorji na Hilbertovih prostorih}
\title{Bounded linear operators on Hilbert spaces}

\mentor{dr.~Daniel~Eremita}

\letnica{2024/25}

%  V povzetku na kratko opišite vsebinske rezultate dela. Sem ne sodi razlaga organizacije dela --
%  v katerem poglavju/razdelku je kaj, pač pa le opis vsebine.
\povzetek{Bo napisan zadnji}

%  Prevod slovenskega povzetka v angleščino.
\abstract{Will be written last}

% navedite vsaj eno klasifikacijsko oznako --
% dostopne so na www.ams.org/mathscinet/msc/msc2020.html
\klasifikacija{47A25, 47B02, 47B15, }
\kljucnebesede{Linearna algebra, funkcionalna analiza, Hilbertov prostor, omejen linearen operator, spekter, spektralni radij, normalen operator, unitaren operator, sebi-adjungiran operator} % navedite nekaj ključnih pojmov, ki nastopajo v delu
\keywords{Linear algebra, funkcional analysis, Hilbert space, bounded linear operator, spectrum, spectral radius, normal operator, unitary operator, self-adjoint operator} % angleški prevod ključnih besed

\zapisiMetaPodatke  % poskrbi za metapodatke in veljaven PDF/A-1b standard

% aktivirajte pakete, ki jih potrebujete
\usepackage{lmodern}
\usepackage{leftidx}
\usepackage{graphicx}
\usepackage{stmaryrd}
\usepackage{wrapfig}
\usepackage{amsthm}
\usepackage{mathrsfs}
\usepackage{silence}
\usepackage{mathtools}
\usepackage{url}
\usepackage{subfigure}
\usepackage{multirow}
\usepackage{lipsum}
\usepackage{wrapfig}
\usepackage{tikz}
\usepackage[format=plain, font=small, labelfont=bf, textfont=it, justification=centerlast]{caption}
\usepackage{booktabs}
\usepackage{siunitx}
\usepackage{enumerate}

% za številske množice uporabite naslednje simbole
\newcommand{\R}{\mathbb{R}}
\newcommand{\F}{\mathbb{F}}
\newcommand{\N}{\mathbb{N}}
\newcommand{\Z}{\mathbb{Z}}
\newcommand{\C}{\mathbb{C}}
\newcommand{\Q}{\mathbb{Q}}
\newcommand{\No}{\N_0}
\newcommand{\n}{\underline{n}}

% matematične operatorje deklarirajte kot take, da jih bo Latex pravilno stavil
% \DeclareMathOperator{\conv}{conv}

% vstavite svoje definicije ...

\newcommand{\abs}[1]{\ensuremath{\lvert #1 \rvert}}
\newcommand{\norm}[1]{\abs{\abs{#1}}}

\newcommand{\Pplus}[1]{\mathbb{#1}_{+}}

\newcommand{\pojem}[1]{\ensuremath{\emph{#1}}}
\newcommand{\con}{\ensuremath{\mathscr{C}}}

\newcommand{\Sp}[2]{\ensuremath{\left<#1, #2\right>}}
\newcommand{\Spp}[2]{\ensuremath{\left<\left<#1, #2\right>\right>}}
\newcommand{\Sppp}[2]{\ensuremath{\left<\left<\left<#1, #2\right>\right>\right>}}


\newcommand{\map}[3]{\ensuremath{{#1}:{#2}\rightarrow{#3}}}
\newcommand{\pra}[3]{{#1}{\ast}({#2}) = {#3}}

\newcommand{\Gen}[1]{\ensuremath{\left<{#1}\right>}}

%===============================================================================
\begin{document}
	%\maketitle je že vključen v class fmfdelo, ki avtomatsko sestavi naslovno stran!
	
	\tableofcontents
	
	\section{Uvod}\label{sect:intro}
	Sem spada uvod, ki bo napisan, ko bo naloga bolj vsebinsko dovršena.
	
	\section{Hilbertovi prostori}
	V tem poglavju bomo najprej osvežili znanje o nekaterih pomembnih definicijah in rezultatih iz študije Banachovih in Hilbertovih prostorov. V nadaljevanju bomo pozornost posvetili t.~i.~ adjungiranim operatorjem, nato pa še njihovim posebnim primerom - normalnim, sebi-adjungiranim in unitarnim operatorjem. Pri tem se bomo primarno sklicevali na vir \cite{bib:Rynne}.
	
		Tukaj bom povedal osnovno o Hilbertovih prostorih: \begin{itemize}
			\item Izrek o odprti preslikavi (brez dokaza)
			\item Kaj so Hilbertovi prostori
			\item Neenakost $C-S-B$
			\item Ortogonalni komplement
			\item Riezsov izrek
			\item Prostor omejenih linearnih operatorjev nad Hilbertovimi prostori
		\end{itemize}
		
		\begin{lema}
			\label{lem:invnorm}
			Za poljubna normirana prostora $X$ ter $Y$ in poljuben $T\in B(X, Y)$ velja: Če je $T$ obrnljiv, je $\norm{Tx}\geq \norm{T^{-1}}^{-1}\norm{x};~\forall x\in X$.
		\end{lema}
		
		\begin{lema}
			\label{lem:banzap}
			Za poljuben Banachov prostor $X$, poljuben normiran prostor $Y$ ter poljuben $T\in B(X, Y)$ velja: Če obstaja tak $r>0$, da je $\norm{Tx}\geq r\norm{x};~\forall x\in X$, je $ImT$ zaprti podprostor v $Y$.
		\end{lema}
		
		\begin{definicija}
			\label{def:ScalProd}
			Naj bo $V$ vektorski prostor nad $\F$. Preslikavi $\map{\Sp{.}{.}}{V\times V}{\R}$ pravimo \pojem{skalarni produkt} na $V$, če velja: \begin{itemize}
				\item $\Sp{x}{x} \geq 0;~\forall x\in V$
				\item $\Sp{x}{x} = 0 \iff x = 0$
				\item $\Sp{x}{y} = \overline{\Sp{y}{x}};~\forall x, y\in V$
				\item $\Sp{\alpha_1\cdot x_1 + \alpha_2\cdot x_2}{y} = \alpha_1\Sp{x_1}{y} +\alpha_2\Sp{x_2}{y};~\forall \alpha_1, \alpha_2\in\F~\&~\forall x_1, x_2, y\in V$
			\end{itemize}
			Če je $V$ vektorski prostor (nad $\F$) in $\Sp{.}{.}$ skalarni produkt na njem, pravimo, da je $(V, +, \cdot, \Sp{.}{.})$ \pojem{prostor s skalarnim produktom}.
		\end{definicija}
		
		\begin{primer}
			\label{prim:ScalProd}
			Spomnimo se nekaj znanih primerov vektorskih prostorov s skalarnim produktom: \begin{itemize}
				\item Prostor $\R^n$ (za $n\in\N$) skupaj s skalarnim produktom $\Sp{x}{y} = \sum_{i = 1}^{n}x_iy_i$.
				\item Prostor $\C^n$ (za $n\in\N$) skupaj s skalarnim produktom $\Sp{z}{w} = \sum_{i = 1}^{n}z_i\overline{w_i}$.
				\item V prostoru zaporedij $l^2$ je s predpisom $\Sp{\bar{x}}{\bar{y}} = \sum_{i = 1}^{\infty}x_i\overline{y_i};~\forall \bar{x}, \bar{y}\in l^2$ definiran skalarni produkt.
				\item V prostoru zveznih funkcij na zaprtem intervalu $[a, b], \mathcal{C}([a, b])$, je s predpisom $\Sp{f}{g} = \int_{a}^{b}f(x)g(x)dx;~\forall f, g \in \mathcal{C}([a, b])$ definiran skalarni produkt.
			\end{itemize}
		\end{primer}
		
		\begin{opomba}
			\label{op:ScalProd}
			Naj bo $(V, +, \cdot, \Sp{.}{.})$ prostor s skalarnim produktom nad poljem $\F$. Enostavno je preveriti, da je s predpisom $\norm{x} = \sqrt{\Sp{x}{x}}~\forall x\in V$ definirana norma na $V$. Za tako normo pravimo, da je \pojem{porojena s skalarnim produktom} $\Sp{.}{.}$.
		\end{opomba}
		
		\begin{definicija}
			\label{def:PolnMP}
			Naj bo $(M, d)$ poljuben metrični prostor. Pravimo, da je $M$ \pojem{poln} metrični prostor, če je vsako Cauchyjevo zaporedje v $M$ konvergentno v $M$.
		\end{definicija}
		
		\begin{primer}
			\label{prim:PolnMP}
			Navedimo nekaj klasičnih primerov polnih metričnih prostorov.
			\begin{itemize}
				\item Prostori $\R^n$ ($n\in\N$) s standardno Evklidsko normo.
				\item Prostori $\C^n$ ($n\in\N$) s standardno Evklidsko normo.
				\item Prostor $l^2$, skupaj z normo, definirano s predpisom $\norm{\bar{x}}_2= \left(\sum_{i=1}^{\infty}\abs{x_i}^2\right)^\frac{1}{2}$; $\forall \bar{x}\in l^2$, je polni metrični prostor.
				\item Prostor s kvadratom integrabilnih funkcij na (realnem ali kompleksnem) merljivem prostoru $X$, opremljenim z Lebesguesovo mero $\mu$; $\mathcal{L}^2(X, \mu)$; skupaj z normo, definirano s predpisom $\norm{f}_2= \left(\int_{X}\abs{f}^2d\mu\right)^\frac{1}{2};\forall f\in \mathcal{L}^2(X, \mu)$.
			\end{itemize}
		\end{primer}
		
		Sedaj definirajmo Hilbertove prostore.
		
		\begin{definicija}
			\label{def:Hilbert}
			Naj bo $(V, \Sp{.}{.})$ poljuben prostor s skalarnim produktom nad poljem $\F$. Naj bo $\norm{.}$ norma na $V$ porojena s $\Sp{.}{.}$ in naj bo $\map{d}{V\times V}{\R}$ metrika na $V$ porojena z $\norm{.}$. Če je $(V, d)$ poln metrični prostor, pravimo, da je $(V, \Sp{.}{.})$ \pojem{Hilbertov prostor} nad $\F$. 
		\end{definicija}
		
		\begin{primer}
			\label{prim:Hilbert}
			Navedimo nekaj znanih primerov Hilbertovih prostorov. \begin{itemize}
				\item Prostori $\R^n$ (za $n\in\N$) skupaj s standardnim skalarnim produktom.
				\item Prostori $\C^n$ (za $n\in\N$) skupaj s (standardnim) skalarnim produktom iz primera \ref{prim:ScalProd}.
				\item Prostor $l^2$ skupaj s skalarnim produktom iz primera \ref{prim:ScalProd}.
				\item Prostor $\mathcal{L}^2(X, \mu)$ iz primera \ref{prim:PolnMP}, skupaj s skalarnim produktom, definiranim s predpisom $\Sp{f}{g}= \int_{[a, b]}(f\cdot \bar{g}) d\mu;\forall f, g\in \mathcal{L}^2(X, \mu)$.
			\end{itemize}
		\end{primer}
		
		\begin{opomba}
			\label{opomb:Hilbert}
			V primeru, ko je merljivi prostor $X$ v primeru \ref{prim:Hilbert} zaprti interval $[a, b]$, sta Lebesguesov in Riemannov integral enaka. Tako oznako za množico poenostavimo na $\mathcal{L}^2([a, b])$, predpis za skalarni produkt pa se spremeni v: $\Sp{f}{g} = \int_{a}^{b}f(x)\overline{g(x)}dx;~\forall f, g \in \mathcal{L}^2([a, b])$.
		\end{opomba}
		
		\begin{posledica}
			\label{posl:zaportoggost}
			Za poljuben zaprti podprostor $Y$ poljubnega Hilbertovega prostora velja $Y^{\bot\bot} = Y$.
		\end{posledica}
		
		\begin{posledica}
			\label{posl:ortoggost}
			Za poljuben podprostor $Y$ poljubnega Hilbertovega prostora velja $Y^{\bot\bot} = \overline{Y}$.
		\end{posledica}
		
		\begin{lema}
			\label{lem:scalprodopenak}
			Naj bo $(X, \Sp{.}{.})$ poljuben kompleksen vektorski prostor s skalarnim produktom in naj bosta $S, T \in B(X)$ poljubna. Če $\forall x\in X$ velja $\Sp{Tx}{x} = \Sp{Sx}{x}$, potem je $T = S$.
		\end{lema}
		
		\begin{izrek}
			\label{izr:Hilbertopinv}
			Naj bo $X$ poljuben kompleksen Hilbertov prostor. Naj bo $T\in B(X)$ poljuben. Če je $\norm{T}< 1$, je operator $I-T$ obrnljiv in velja: $(I-T)^{-1} = \sum_{n = 1}^{\infty}T^n$.
		\end{izrek}
		
		\begin{izrek}
			\label{izr:Hilbertinvset}
			Naj bo $X$ poljuben kompleksen Hilbertov prostor ter z $\mathcal{G}(X)$ označimo množico vseh obrnljivih operatorjev v $B(X)$. Velja: \begin{enumerate}[a)]
				\item $(\mathcal{G}, \cdot)$ je grupa.
				\item Množica $\mathcal{G}$ je odprta v $B(X)$.
				\item Preslikava invertiranja, $\map{^{-1}}{\mathcal{G}}{\mathcal{G}}$, s predpisom $^{-1}(T) = T^{-1};~\forall T\in \mathcal{G}(X)$ je zvezna.
			\end{enumerate}
		\end{izrek}
		
	\section{Omejeni linearni operatorji nad Hilbertovimi prostori}
		V tem poglavju bomo obravnavali lastnosti omejenih linearnih operatorjev nad Hilbertovimi prostori ter lastnosti njihovih spektrov, kadar sta domena in kodomena kompleksna Hilbertova prostora. Posebej nas bo zanimala operacija adjungiranja, zato bomo najprej povedali nekaj o njej.
		\subsection{Adjungirani operatorji}
		
			\begin{izrek}
				\label{izr:Adjexist}
				Naj bosta $(X, \Spp{.}{.})$ in $(Y, \Sp{.}{.})$ poljubna kompleksna Hilbertova prostora in naj bo $T$ poljuben element $B(X, Y)$. Tedaj obstaja enolično določen operator $T^*\in B(Y, X)$, za katerega velja $$\Sp{Tx}{y} = \Spp{x}{T^*y};~\forall x\in X~\&~\forall y \in Y $$
			\end{izrek}
			
			\begin{dokaz}
				Naj bo $y\in Y$ poljuben ter naj bo $\map{f_y}{X}{\C}$ preslikava definirana s predpisom $f_y(x) = \Sp{Tx}{y};~\forall x\in X$. Opazimo, da je $f_y$ linearen funkcional nad $X$. Po neenakosti Cauchy-Schwarz-Bunyakowsky potem velja: $$\abs{f_y(x)} = \abs{\Sp{Tx}{y}} \leq \norm{Tx}_Y \cdot \norm{y}_Y \leq \norm{T}\cdot\norm{x}_X\cdot\norm{y}_Y;~ \forall x\in X$$ To pomeni, da je $f_y$ omejen linearen funkcional nad $X$. Posledično velja, po Riezsovem izreku, da obstaja enolično določen $z\in X$, tak, da je $f_y(x) = \Spp{x}{z};~\forall x\in X$. Sedaj definiramo preslikavo $\map{T^*}{Y}{X}$, ki vsakemu $y\in Y$ priredi pripadajoči $z\in X$ (tisti, enolično določen, za katerega je $f_y(x) = \Spp{x}{z};~\forall x\in X$). Posledično velja: $$\Sp{Tx}{y}=\Spp{x}{T^*y}~\forall x\in X,~\forall y\in Y$$
				Sedaj preverimo, da je $T^*$ omejen linearen operator. Naj bodo $y_1, y_2 \in Y$ in $x\in X$ poljubni elementi ter $\alpha, \beta \in \C$ poljubna skalarja. Tedaj je: \begin{align*}
					\Spp{x}{T^*(\alpha y_1 + \beta y_2)} &= \Sp{Tx}{\alpha y_1 + \beta y_2} \\
					&= \bar{\alpha}\Sp{Tx}{y_1} + \bar{\beta}\Sp{Tx}{y_2} \\
					&= \bar{\alpha}\Spp{x}{T^*y_1} + \bar{\beta}\Spp{x}{T^*y_2} \\
					&= \Spp{x}{\alpha T^*y_1} + \Spp{x}{\beta T^*y_2}
				\end{align*}
				Sledi, da je $T^*(\alpha y_1 + \beta y_2)=\alpha T^*y_1 + \beta T^*y_2;~\forall x\in X, \forall y_1, y_2 \in Y, \forall \alpha, \beta \in \C$, torej je $T^*$ res linearna preslikava. Dodatno, velja: $$\norm{T^*y}_X^2 = \Spp{T^*y}{T^*y} = \Sp{TT^*y}{y} \leq \norm{TT^*y}_Y\norm{y}_Y \leq \norm{T}\norm{T^*y}_X\norm{y}_Y;~\forall y\in Y$$
				Če je $\norm{T^*y}_X \neq 0$, dobljeno neenakost delimo z $\norm{T^*y}_X$ ter tako dobimo oceno $\norm{T^*y}_X \leq \norm{T}\norm{y}_Y$. V primeru, ko je $\norm{T^*y}_X = 0$, prejšnja ocena velja trivialno. Sledi, da je $$\norm{T^*y}_X \leq \norm{T}\norm{y}_Y; \forall y\in Y$$
				Preslikava $T^*$ je torej omejen linearen operator in velja $\norm{T^*}\leq \norm{T}$. Za konec še pokažimo, da je $T^*$ enolično določen. Denimo, da imamo $U_1, U_2 \in B(Y, X)$, za katera velja: $\Sp{Tx}{y} = \Spp{x}{U_1 y} = \Spp{x}{U_2 y};~\forall x\in X, \forall y\in Y$. Tedaj velja, da je $U_1y = U_2y;~\forall y\in Y$, torej je $U_1 = U_2$. Sledi, da je $T^*$ res enolično določen.
			\end{dokaz}
			Izrek \ref{izr:Adjexist} nas motivira, da vpeljemo naslednjo definicijo.
		
			\begin{definicija}
				\label{def:adjop}
				Naj bosta $X$ in $Y$ kompleksna Hilbertova prostora ter naj bo $T\in B(X, Y)$. Tedaj operatorju $T^*$ iz izreka \ref{izr:Adjexist} pravimo \pojem{adjungiran operator} operatorja $T$.
			\end{definicija}
			
			\begin{primer}
				\label{prim:adjop1} Spomnimo se operatorja $D\in B(l^2)$, imenovanega ">desni premik"<, s predpisom $D(x_1, x_2, x_3, \ldots) = (0, x_1, x_2, \ldots);~\forall \bar{x}\in l^2$ in določimo predpis njegovega adjungiranega operatorja $D^*$. Naj bosta $\bar{x}, \bar{y}\in l^2$ poljubna ter označimo $\bar{z} = D^*y$. Tedaj po definiciji adjungiranega operatorja velja, da je $\Sp{Dx}{y} = \Sp{x}{D^*y} = \Sp{x}{z}$ oziroma $$\Sp{(0, x_1, x_2, \ldots)}{(y_1, y_2, y_3, \ldots)}=\Sp{(x_1, x_2, x_3, \ldots)}{(z_1, z_2, z_3, \ldots)}$$
				Ko to enakost razpišemo, dobimo: $$x_1\overline{y_2}+x_2\overline{y_3} + \ldots = x_1\overline{z_1} + x_2\overline{z_2}+x_3\overline{z_3} + \ldots$$
				Opazimo, da če velja $z_1 = y_2, z_2 = y_3, \ldots z_i = y_{i+1}\forall i\in \N$, potem bo naša enačba veljavna za poljuben $\bar{x}$. Enoličnost adjungiranega operatorja nam posledično da predpis: $D^*(y_1, y_2, y_3 \ldots) = (y_2, y_3, y_4, \ldots)$; $\forall \bar{y}\in l^2$. Še več, prepoznamo, da je dobljeni operator ravno t.~i.~ ">levi premik"<, ki ga tipično označimo z $L$.
			\end{primer}
			
			\begin{primer}
				\label{prim:adjop2} Za vsak $k\in \mathcal{C}([0, 1])$ definiramo operator $T_k\in B(\mathcal{L}^2([0, 1]))$ (skupaj s skalarnim produktom iz primera \ref{prim:Hilbert}) s predpisom $(T_kg)(t) = k(t)g(t);~\forall t\in [0, 1]$ (oziroma $T_kg = kg$), $\forall g\in \mathcal{L}^2([0, 1])$. Določimo predpis adjungiranega operatorja $(T_k)^*$.
				
				Naj bosta $f, g \in \mathcal{L}^2([0, 1])$ poljubna ter naj bo $h = (T_k)^*g$. Tedaj, po definiciji adjungiranega operatorja, velja: $$\Sp{T_kf}{g} = \Sp{f}{(T_k)^*g} = \Sp{f}{h}$$
				V skladu z opombo \ref{opomb:Hilbert} potem sledi: $$\int_{0}^{1}k(t)f(t)\overline{g(t)}dt = \int_{0}^{1}f(t)\overline{h(t)}dt$$
				Ta enačba bo držala, če je $\overline{h(t)} = k(t)\overline{g(t)};~\forall t\in [0, 1]$ oziroma $h(t) = \overline{k(t)}g(t);~\forall t\in [0, 1]$. Drugače povedano, enačba velja, če je $h = T_{\overline{k}}g$. Ker sta bila $f$ in $g$ poljubna, sledi, da sklep velja za vsak par elementov $\mathcal{L}^2([0, 1])$. Ker je adjungirani operator enolično določen, sledi, da je $(T_k)^* = T_{\overline{k}}$.
			\end{primer}
			
			Sedaj bomo navedli in dokazali nekatere pomembne lastnosti adjungiranja.
			
			\begin{izrek}
				\label{izr:adjlast}
				Naj bodo $(X, \Spp{.}{.})$, $(Y, \Sp{.}{.})$ in $(Z, \Sppp{.}{.})$ poljubni kompleksni Hilbertovi prostori ter naj bodo $U, V \in B(X, Y)$ ter $T \in B(Y, Z)$ poljubni omejeni linearni operatorji. Tedaj velja: \begin{enumerate}[a)]
					\item $(\alpha U + \beta V)^* = \bar{\alpha}U^* + \bar{\beta}V^*; \forall \alpha, \beta \in \C$.
					\item $(TU)^* = U^*T^*$.
					\item $(U^*)^* = U$.
					\item $\norm{U^*} = \norm{U}$.
					\item Preslikava $\map{f}{B(X, Y)}{B(Y, X)}$, definirana s predpisom $f(U) = U^*$ je zvezna.
					\item $\norm{U^*U} = \norm{U}^2$.
				\end{enumerate}
			\end{izrek}
			
			\begin{dokaz}
				\begin{enumerate}[a)]
					\item Naj bodo $U, V \in B(X, Y)$ in $\alpha, \beta \in \C$ poljubni. Tedaj $\forall x\in X, \forall y\in Y$ velja: \begin{align*}
						\Spp{x}{(\alpha U + \beta V)^*y} &= \Sp{(\alpha U + \beta V)x}{y} \\
						&= \alpha \Sp{Ux}{y} + \beta \Sp{Vx}{y} \\
						&= \alpha \Spp{x}{U^*y} + \beta\Spp{x}{V^*y} \\
						&= \Spp{x}{\bar{\alpha}U^*y} + \Spp{x}{\bar{\beta}V^*y} \\
						&= \Spp{x}{(\bar{\alpha} U^* + \bar{\beta} V^*)y}
					\end{align*} 
					Sledi, da je $(\alpha U + \beta V)^* = (\bar{\alpha} U^* + \bar{\beta} V^*)$.
					\item Naj bosta $U\in B(X, Y)$ in $T\in B(Y, Z)$ poljubna. Tedaj $\forall x\in X, \forall z\in Z$ velja: $$\Spp{x}{(TU)^*z} = \Sppp{(TU)x}{z} = \Sppp{T(Ux)}{z} = \Sp{Ux}{T^*z} = \Spp{x}{U^*T^*z}$$
					Posledično je res $(TU)^* = U^*T^*$.
					\item Naj bo $U\in B(X, Y)$ poljuben. Tedaj $\forall x\in X, \forall y\in Y$ velja: $$\Sp{y}{(U^*)^*x} = \Spp{U^*y}{x} = \overline{\Spp{x}{U^*y}} = \overline{\Sp{Ux}{y}} = \Sp{y}{Ux}$$
					Posledično je $(U^*)^* = U$.
					\item V dokazu izreka \ref{izr:Adjexist} smo že pokazali, da je $\norm{U^*} \leq \norm{U}$. Ko upoštevamo prejšnjo točko, vidimo, da velja: $$\norm{U} = \norm{(U^*)^*} \leq \norm{U^*} \leq \norm{U}$$
					Posledično sledi, da je $\norm{U^*} = \norm{U}$.
					\item Naj bodo $U, V \in B(X, Y)$ $\varepsilon > 0$ poljubni in izberemo $\delta = \varepsilon$. Denimo, da je $\norm{U - V} < \delta$. Tedaj je $\norm{f(U) - f(V)} = \norm{U^* - V^*} = \norm{(U - V)^*}$. Po prejšnji točki je $\norm{(U-V)^*} = \norm{U - V} < \delta = \varepsilon$, torej je $f$ zvezna.
					\item Naj bo $U\in B(X, Y)$ poljuben. Ker že vemo, da je $\norm{U^*} = \norm{U}$, hitro vidimo, da je $\norm{U^*U}\leq\norm{U^*}\norm{U} = \norm{U}^2$.
					Hkrati vidimo, da za $\forall x\in X$ velja $\norm{Ux}_Y^2 = \Sp{Ux}{Ux} = \Spp{U^*Ux}{x}$. Ko upoštevamo neenakost Cauchy-Schwarz-Bunyakowsky, dobimo oceno $\Spp{U^*Ux}{x} \leq \norm{U^*Ux}_X\norm{x}_X \leq \norm{U^*U}\norm{x}_X^2$. Sledi ocena: $\norm{Ux}_Y^2 \leq \norm{U^*U}\norm{x}_X^2 \forall x\in X$, torej je $\norm{U}^2 \leq \norm{U^* U}$. Posledično velja iskana enakost.
				\end{enumerate}
			\end{dokaz}
			
			Pri obravnavi adjungiranih operatorjev bomo potrebovali naslednji pomožni rezultat.
			
			\begin{lema}
				\label{lem:adjlast2}
				Naj bosta $(X, \Spp{.}{.})$ in $(Y, \Sp{.}{.})$ poljubna kompleksna Hilbertova prostora ter naj bo $T\in B(X, Y)$ poljuben. Tedaj velja: \begin{enumerate}[a)]
					\item $KerT = (ImT^*)^{\bot}$.
					\item $KerT^* = (ImT)^{\bot}$.
					\item $KerT^* = \{0\} \iff ImT$ je gosta v $Y$.
				\end{enumerate}
			\end{lema}
			
			\begin{dokaz}
				\begin{enumerate}[a)]
					\item Najprej pokažimo, da je $KerT \subseteq (ImT^*)^\bot$. Naj bo $x\in KerT$ poljuben ter izberemo poljuben $z\in ImT^*$. Potem za $z$ obstaja $y\in Y$, da je $T^*y = z$. Posledično, je $$\Spp{x}{z} = \Spp{x}{T^*y} = \Sp{Tx}{y} = \Sp{0}{y} = 0$$
					Sledi, da je $x\in (ImT^*)^\bot$. Ker je $x$ bil poljuben, premislek velja za vsak $x\in KerT$, torej je $KerT \subseteq (ImT^*)^\bot$.
					Naj bo sedaj $x\in (ImT^*)^\bot$ poljuben. Ker je $T^*Tx \in ImT^*$, velja: $$\norm{Tx}^2 = \Sp{Tx}{Tx} = \Sp{x}{T^*Tx}= 0$$
					Sledi, da je $Tx = 0$ oz. $x\in KerT$. Ker je bil $x$ poljuben, sledi $(ImT^*)^\bot \subseteq KerT$. Posledično velja enakost.
					\item Da dokažemo ta rezultat, upoštevamo prejšnjo točko tega dokaza za $T^*$ ter točko $c)$ izreka \ref{izr:adjlast}. Posledično je $KerT^* = (Im(T^*)^*)^\bot = (ImT)^\bot$.
					\item Da dokažemo ta rezultat bomo dokazali, da veljata implikaciji v obe smeri. \begin{itemize}
						\item[$\Rightarrow)$] Denimo, da je $KerT^* = \{0\}$. Potem vidimo, upoštevajoč posledico \ref{posl:ortoggost} ter točko $a)$ tega dokaza, da velja: $$\overline{ImT} = ((ImT)^\bot)^\bot = (KerT^*)^\bot = \{0\}^\bot = Y$$
						Sledi, da je $ImT$ gosta v $Y$ po definiciji.
						\item[$\Leftarrow)$] Denimo, da je $ImT$ gosta v $Y$. Potem po posledici \ref{posl:ortoggost} velja, da je $$((ImT)^{\bot})^\bot = \overline{ImT} = Y$$ Po točki $b)$ tega dokaza velja $KerT^* = (ImT)^\bot$. Ker je $ImT\subseteq Y$ neprazna podmnožica, velja, da je $(ImT)^\bot$ zaprti podprostor v $Y$. Po posledici \ref{posl:zaportoggost} potem sledi: $$KerT^* = (ImT)^\bot =(((ImT)^\bot)^\bot)^\bot = Y^\bot = \{0\}$$
					\end{itemize} 
					
				\end{enumerate}
			\end{dokaz}

			\begin{posledica}
				\label{posl:adjinv}
				Naj bo $X$ poljuben kompleksen Hilbertov prostor ter naj bo $T\in B(X)$. Tedaj je operator $T$ obrnljiv natanko tedaj, ko je $KerT^* = \{0\}$ in obstaja tak $r > 0$, da je $\norm{Tx} \geq r\norm{x}~\forall x\in X$.
			\end{posledica}
			
			\begin{dokaz}
				Dokazali bomo obe implikaciji.
				\begin{itemize}
					\item[$\Rightarrow)$] Denimo, da je $T$ obrnljiv. Tedaj je $T$ bijekcija in njegov inverz, $T^{-1}$, pripada $B(X)$. Posledično je $ImT = Y$. Ker je $X$ Hilbertov prostor, je tudi normiran prostor, torej po lemi \ref{lem:invnorm} velja, da je $\norm{Tx}\geq \norm{T^{-1}}^{-1}\norm{x};~\forall x\in X$. Določimo $r = \norm{T^{-1}}^{-1}$. Ker je $X$ tudi Banachov prostor, po lemi \ref{lem:banzap} velja, da je $ImT$ zaprti podprostor v $Y$. Sledi, da je $ImT = \overline{ImT} = Y$, torej je $ImT$ gost v $Y$. Po lemi \ref{lem:adjlast2} je potem $KerT^* = \{0\}$.
					\item[$\Leftarrow)$] Denimo sedaj, da je $KerT^* = \{0\}$ in da obstaja tak $r>0$, da za $\forall x\in X$ velja $\norm{Tx} \geq r\norm{x}$. Po lemi \ref{lem:adjlast2} potem sledi, da je $ImT$ gost v $Y$. Lema \ref{lem:banzap} nam tudi pove, da je $ImT$ zaprti podprostor v $Y$. Sledi, da je $ImT = \overline{ImT} = Y$.
					Naj bo sedaj $x\in KerT$ poljuben. Tedaj je $Tx = 0$, po predpostavki pa potem sledi, da je $0 = \norm{Tx} \geq r\norm{x}$. To je pa možno le, kadar je $x = 0$. Sledi, da je $KerT = \{0\}$, torej je $T$ bijektiven endomorfizem nad Banachovim prostorom $X$. Po izreku o odprti preslikavi sledi, da je $T$ obrnljiv.
				\end{itemize}
			\end{dokaz}
			
			\begin{primer}
				\label{prim:adjinv}
				Primer neobrnljivega operatorja najdemo v desnem premiku $D\in B(l^2)$. Kot smo premislili v primeru \ref{prim:adjop2}, je njegov adjungirani operator ravno levi premik $L\in B(l^2)$, za katerega pa velja, da je $L(1, 0, 0, \ldots) = \bar{0}$. Sledi, da $KerD^* = KerL \neq \{0\}$, torej $D$ ni obrnljiv.
			\end{primer}
			
			Za konec tega podpoglavja premislimo, v sledeči lemi, kako sta povezani operaciji invertiranja in adjungiranja.
			
			\begin{lema}
				\label{lem:adjinv}
				Naj bo $X$ poljuben kompleksen Hilbertov prostor ter naj bo $T\in B(X)$ poljuben. Če je $T$ obrnljiv, je obrnljiv tudi $T^*$.
			\end{lema}
		
			\begin{dokaz}
				Naj bo $T\in B(X)$ poljuben obrnljiv operator. Tedaj velja, da je $TT^{-1} = T^{-1}T = I$. Enačbo adjungiramo ter tako dobimo $(TT^{-1})^* = (T^{-1}T)^* = I^*$. Ko upoštevamo točko $b)$ izreka \ref{izr:adjlast}, se enačba poenostavi v: $$(T^{-1})^*T^* = T^*(T^{-1})^* = I$$
				Posledično je $T^*$ obrnljiv ter $(T^*)^{-1} = (T^{-1})^*$.
			\end{dokaz}
			
		\subsection{Normalni operatorji}
		
			Sedaj se bomo posvetili, glede na adjungiranje, posebnim primerom operatorjev. Prvi izmed teh, so t.~i.~ normalni operatorji.
			
			\begin{definicija}
				\label{def:normalop}
				Naj bo $X$ poljuben kompleksen Hilbertov prostor. Pravimo, da je $T \in B(X)$ \pojem{normalen}, če je $TT^* = T^*T$.
			\end{definicija}
			
			\begin{primer}
				\label{prim:normalop}
				Enostavno je videti, da je operator $T_k$, kot je bil definiran v primeru \ref{prim:adjop2}, normalen operator. Naj bo $f\in \mathcal{L}^2([0, 1])$ poljuben.
				\begin{align*}
					(T_k(T_k)^*)f &= (T_kT_{\overline{k}})f =T_k(\overline{k}f) = k\overline{k}f \\
					((T_k)^*T_k)f &= (T_{\overline{k}}T_k)f = T_{\overline{k}}(kf)=\overline{k}kf = k\overline{k}f
				\end{align*}
				Ker je bil $f$ poljuben, sklep velja za vse elemente $\mathcal{L}^2([0, 1])$. Sledi, da je $T_k(T_k)^* = (T_k)^*T_k$, torej je operator $T_k$ res normalen.
			\end{primer}
			
			\begin{lema}
				\label{lem:normalnorm}
				Naj bo $(X, \Sp{.}{.})$ poljuben kompleksen Hilbertov prostor. Naj bo $T\in B(X)$ poljuben normalen operator ter $r > 0$ poljuben. Velja: \begin{enumerate}[a)]
					\item $\norm{Tx} = \norm{T^*x};~\forall x\in X$.
					\item Če je $\norm{Tx} \geq r\norm{x};~\forall x\in X$, je $KerT^* = \{0\}$.
				\end{enumerate}
			\end{lema}
			
			\begin{dokaz}
				\begin{enumerate}[a)]
					\item Naj bo $x\in X$ poljuben. Potem je \begin{align*}
						\norm{Tx}^2 - \norm{T^*x}^2 &= \Sp{Tx}{Tx} - \Sp{T^*x}{T^*x} = \Sp{T^*Tx}{x} - \Sp{TT^*x}{x}\\
						&= \Sp{T^*Tx - TT^*x}{x} = \Sp{(T^*T - TT^*)x}{x}
					\end{align*}
					Ker je $T$ normalen, je $TT^* = T^*T$, torej je $$\norm{Tx}^2 - \norm{T^*x}^2 = \Sp{(T^*T - TT^*)x}{x} = \Sp{0}{x} = 0 $$
					Posledično je $\norm{Tx} - \norm{T^*x}$. Ker je bil $x$ poljuben, enakost velja za $\forall x\in X$.
					\item Denimo, da obstaja tak $r>0$, da je $\norm{Tx} \geq r\norm{x};~\forall x\in X$. Naj bo $x\in KerT^*$ poljuben. Po točki $a)$ tega dokaza je potem $0 = \norm{T^*x} = \norm{Tx} \geq r\norm{x}$. Sledi, da je $\norm{x} = 0$ oz. $x = 0$. Posledično je $KerT^* = \{0\}$.
				\end{enumerate}
			\end{dokaz}
			
			Na podlagi posledice \ref{posl:adjinv} in leme \ref{lem:normalnorm} sledi naslednja posledica.
			
			\begin{posledica}
				\label{posl:normalinv}
				Naj bo $X$ poljuben kompleksen Hilbertov prostor in naj bo $T\in B(X)$ poljuben normalen operator. Tedaj je operator $T$ obrnljiv natanko tedaj, ko obstaja tak $r > 0$, da je $\norm{Tx} \geq r\norm{x}~\forall x\in X$.
			\end{posledica}
			
			\begin{dokaz}
				Da dokažemo ekvivalenco, bomo dokazali implikaciji v obe smeri.
				\begin{itemize}
					\item[$\Rightarrow)$] Velja direktno po posledici \ref{posl:adjinv}.
					\item[$\Leftarrow)$] Denimo, da obstaja tak $r>0$, da je $\norm{Tx} \geq r\norm{x}~\forall x\in X$. Po lemi \ref{lem:normalnorm} je potem $KerT^* = \{0\}$. Po posledici \ref{posl:adjinv} je potem $T$ obrnljiv.
				\end{itemize}
			\end{dokaz}
			
			\subsection{Sebi-adjungirani operatorji}
			
			Pri obravnavi preslikav nas pogosto zanimajo t.~i.~ ">fiksne točke"< - točke, ki jih preslikava preslika nazaj vase. V primeru adjungiranja, operatorjem, ki jih operacija ">fiksira"< damo posebno ime.
			
			\begin{definicija}
				\label{def:sebiadjop}
				Naj bo $X$ poljuben kompleksen Hilbertov prostor. Pravimo, da je $T\in B(X)$ \pojem{sebi-adjungiran}, če velja $T = T^*$.
			\end{definicija}
			
			\begin{primer}
				\label{prim:sebiadjop}
				Trivialno je videti, da je identični operator $I$ sebi-adjungiran. Dodatno, če se vrnemo k operatorju $T_k$ iz primera \ref{prim:adjop2}, vidimo, da je tudi ta sebi-adjungiran za vsako izbiro funkcije $k\in\mathcal{C}([0, 1])$. To je res, saj je funkcija $k\in\mathcal{C}([0, 1])$ realna, torej je $\overline{k} = k$ in posledično je $(T_k)^* = T_{\overline{k}} = T_k$.
				
			\end{primer}
			
			V naslednji lemi so povzete nekatere osnovne lastnosti sebi-adjungiranih operatorjev.
			
			\begin{lema}
				\label{lem:sebiadjlast}
				Naj bo $X$ poljuben kompleksen Hilbertov prostor. Naj bo $\mathcal{S}(X)$ množica vseh sebi-adjungiranih operatorjev znotraj $B(X)$ in naj bo $T\in B(X)$ poljuben. Velja:
				\begin{enumerate}[a)]
					\item Za poljubni $\alpha, \beta \in \R$ in poljubna $T_1, T_2 \in \mathcal{S}(X)$ velja: $\alpha T_1 + \beta T_2 \in \mathcal{S}(X)$.
					\item Množica $\mathcal{S}(X)$ je zaprta podmnožica $B(X)$.
					\item Operatorja $TT^*$ in $T^*T$ sta sebi-adjungirana.
					\item $T = R + iS$, za neka sebi-adjungirana operatorja $R, S \in \mathcal{S}(X)$.
				\end{enumerate}
			\end{lema}
			
			\begin{dokaz}
				\begin{enumerate}[a)]
					\item Naj bosta $T_1, T_2 \in \mathcal{S}(X)$ poljubna. Ker sta operatorja sebi-adjungirana, po prvi točki izreka \ref{izr:adjlast} velja: $(\alpha T_1 + \beta T_2)^* = \alpha T_1^* + \beta T_2^* = \alpha T_1 + \beta T_2$. Posledično je $\alpha T_1 + \beta T_2 \in \mathcal{S}(X)$.
					\item Naj bo $\{T_n\}_{n\in\N}$ konvergentno zaporedje s členi v $\mathcal{S}(X)$ in limito $T\in B(X)$. Predzadnja točka izreka \ref{izr:adjlast} nam pove, da je adjungiranje zvezno, torej je tudi zaporedje $\{T_n^*\}_{n\in\N}$ konvergentno z limito $T^* \in B(X)$. Ker so $T_n \in \mathcal{S}(X)~ \forall n\in \N$, velja $T_n^* = T_n~\forall n\in \N$, torej je $T^* = T$. Posledično je tudi $T\in \mathcal{S}(X)$. Sledi, da je $\mathcal{S}(X)$ zaprta.
					\item Naj bo $T\in B(X)$ poljuben. Tedaj je, upoštevajoč točko $c)$ izreka \ref{izr:adjlast}, $(T^*T)^* = T^*T^{**} = T^*T$. Sledi, da je $T^*T\in \mathcal{S}(X)$ in na enak način premislimo, da enako velja za $TT^*$.
					\item Naj bo $T\in B(X)$ poljuben. Definiramo operatorja $R$ in $S$ na naslednji način: $$R = \frac{1}{2}(T + T^*),~ S=\frac{1}{2i}(T - T^*)$$
					Očitno velja, da je $T = R + iS$. Preverimo še, da sta $R$ in $S$ res sebi-adjungirana.
					\begin{align*}
						R^* &= (\frac{1}{2}(T + T^*))^* = \frac{1}{2}(T^* + T^{**}) = \frac{1}{2}(T^* + T) = R \\
						S^* &= (\frac{1}{2i}(T - T^*))^* = \frac{-1}{2i}(T - T^*)^* = \frac{-1}{2i}(T^* - T^{**}) =\frac{1}{2i}(T - T^*) = S
					\end{align*}
				\end{enumerate}
			\end{dokaz}
			
			\begin{trditev}
				\label{trd:sebiadjop}
				Naj bo $(X, \Sp{.}{.})$ poljuben kompleksen Hilbertov prostor in naj bo $T\in B(X)$ poljuben. Operator $A$ je sebi-adjungiran natanko tedaj, ko je $\Sp{Tx}{x}\in\R;\forall x\in X$.
			\end{trditev}
			
			\begin{dokaz}
				Dokazali bomo implikaciji v obe smeri.
				\begin{itemize}
					\item[$\Rightarrow)$] Denimo, da je $T$ sebi-adjungiran in naj bo $x\in X$ poljuben. Tedaj je $\Sp{Tx}{x} = \Sp{x}{Tx} = \overline{\Sp{Tx}{x}}$, torej je $\Sp{Tx}{x}\in \R$. Ker je bil $x$ poljuben, sklep velja za vse elemente $X$.
					\item[$\Leftarrow)$] Denimo, da je $\Sp{Tx}{x}\in\R;~\forall x\in X$. Tedaj za poljubne $\alpha\in\C, x, y\in X$ velja, da je $\Sp{T(x + \alpha y)}{x + \alpha y}\in \R$ oziroma $$\Sp{Tx}{x} + \alpha\Sp{Ty}{x} + \overline{\alpha}\Sp{Tx}{y} + \alpha\overline{\alpha}\Sp{Ty}{y}\in\R$$
					Prvi in zadnji člen sta, po predpostavki, avtomatsko elementa $\R$. Sledi torej, da je tudi $\alpha\Sp{Ty}{x} + \overline{\alpha}\Sp{Tx}{y}\in\R$. Velja: $$\alpha\Sp{Ty}{x} + \overline{\alpha}\Sp{Tx}{y} = \overline{\alpha}\overline{\Sp{Ty}{x}} + \overline{\overline{\alpha}}\overline{\Sp{Tx}{y}} = \overline{\alpha}\Sp{x}{Ty} + \alpha\Sp{y}{Tx}$$
					Dodatno upoštevamo, da je $\overline{\alpha}\Sp{x}{Ty} + \alpha\Sp{y}{Tx} = \overline{\alpha}\Sp{T^*x}{y} + \alpha\Sp{T^*y}{x}$. Sklepamo torej, da je $\alpha\Sp{Ty}{x} + \overline{\alpha}\Sp{Tx}{y}=\overline{\alpha}\Sp{T^*x}{y} + \alpha\Sp{T^*y}{x}$ za vsak $\alpha\in\C$. Če vstavimo $\alpha = 1$ in $\alpha = i$, dobimo naslednji sistem enačb: \begin{align*}
						\alpha = 1&:\Sp{Tx}{y} + \Sp{Ty}{x} = \Sp{T^*x}{y} + \Sp{T^*y}{x} \\
						\alpha = i&:-i\Sp{Tx}{y} + i\Sp{Ty}{x} = -i\Sp{T^*x}{y} + i\Sp{T^*y}{x}
					\end{align*}
					Prvo enačbo pomnožimo z $i$, jo odštejemo od druge, ter tako dobimo enakost $\Sp{Tx}{y} = \Sp{T^*x}{y}$. Ker sta bila $x$ in $y$ poljubna, ta sklep velja za vse pare $x, y \in X$, torej po lemi \ref{lem:scalprodopenak} sledi, da je $T = T^*$.
				\end{itemize}	
			\end{dokaz}
			
			\subsection{Unitarni operatorji}
			
			Zadnji posebni primer operatorjev, glede na adjungiranje, je primer, v katerem je adjungirani operator hkrati tudi inverz.
			
			\begin{definicija}
				\label{def:unitarop}
				Naj bo $X$ poljuben kompleksen Hilbertov prostor. Pravimo, da je operator $T\in B(X)$ \pojem{unitaren}, če velja: $TT^* = T^*T = I$.
			\end{definicija}
			
			\begin{primer}
				\label{prim:unitarop}
				Za funkcijo $k\in\mathcal{C}([0, 1])$ za katero velja, da je $\abs{k(t)} = 1,~\forall t\in [0, 1]$, je operator $T_k$ iz primera \ref{prim:adjop2} unitaren operator.
				Iz primera \ref{prim:normalop} že vemo, da je za poljuben $k\in\mathcal{C}([0, 1])$ operator $T_k$ normalen ter da je $T_k(T_k)^*f = k\overline{k}f =\abs{k}^2f= (T_k)^*T_kf,~\forall f\in\mathcal{L}^2([0, 1])$. Naj bo torej $f\in\mathcal{L}^2([0, 1])$ poljuben in naj bo $k\in\mathcal{C}([0, 1])$ tak, da je $\abs{k(t)}=1;\forall t\in[0, 1]$. Tedaj je $\abs{k(t)}^2f(t) = f(t);~\forall t\in[0, 1]$, torej je $T_k(T_k)^*f = f = (T_k)^*T_kf$. Ker je bil $f$ poljuben, sklep velja za vse elemente $\mathcal{L}^2([0, 1])$. Sledi, da je $T_k(T_k)^* = I = (T_k)^*T_k$.
			\end{primer}
			
			Naslednji izrek nam pove, da so unitarni operatorji hkrati tudi izometrije.
			
			\begin{izrek}
				\label{izr:uniizo}
				Naj bo $(X, \Sp{.}{.})$ poljuben kompleksen Hilbertov prostor. Naj bosta $T, U\in B(X)$ poljubna. Velja: \begin{enumerate}[a)]
					\item $T^*T = I \iff T$~je izometrija.
					\item $U$ je unitaren $\iff U$~je surjektivna izometrija na $X$.
				\end{enumerate}
			\end{izrek}
			
			\begin{dokaz}
				Obe ekvivalenci bomo dokazali tako, da bomo pokazali, da veljata implikaciji v obe smeri.
				\begin{enumerate}[a)]
					\item Pri dokazu si bomo pomagali z lemo \ref{lem:scalprodopenak}. \begin{itemize}
						\item[$\Rightarrow)$] Denimo, da je $T^*T = I$ in naj bo $x\in X$ poljuben. Tedaj je: $$\norm{Tx}^2=\Sp{Tx}{Tx} = \Sp{T^*Tx}{x}= \Sp{Ix}{x} = \Sp{x}{x} = \norm{x}^2$$
						Ker je bil $x$ poljuben, zgornja enakost velja za vse elemente $X$, torej je $T$ res izometrija.
						\item[$\Leftarrow)$] Denimo, da je $T$ izometrija. Naj bo $x\in X$ poljuben. Tedaj velja $$\Sp{T^*Tx}{x} = \Sp{Tx}{Tx} = \norm{Tx}^2 = \norm{x}^2 = \Sp{Ix}{x}$$
						Ker je $x$ poljuben, sklep velja za vse elemente $X$, torej po lemi \ref{lem:scalprodopenak} sledi, da je $T^*T = I$.
					\end{itemize}
					\item Pri dokazu te ekvivalence, si bomo pomagali s prejšnjo točko dokaza.  \begin{itemize}
						\item[$\Rightarrow)$] Denimo, da je $U$ unitaren. Potem sledi, da je $UU^*=U^*U = I$ in po točki $a)$ je potem $U$ izometrija. Dodatno, naj bo $x\in X$ poljuben. Potem je $x = U(U^*x)$, torej je $x\in ImU$. Sledi, da je $ImU = X$ oz. $U$ je surjektiven.
						\item[$\Leftarrow)$] Denimo, da je $U$ surjektivna izometrija na $X$. Ker je $U$ izometrija, po točki $a)$ tega dokaza sledi, da je $U^*U = I$. Naj bo sedaj $y\in X$ poljuben. Ker je $U$ surjektiven operator na $X$, obstaja tak $x\in X$, da je $Ux = y$. Posledično velja naslednje: $$UU^*y = UU^*(Ux) = U(U^*Ux) = U(Ix) = Ux = y$$
						Ker je bil $y$ poljuben, prejšnji sklep velja za vsak element $X$. Sledi, da je $UU^* = I$, torej je $U$ unitaren operator.
					\end{itemize}
				\end{enumerate}
			\end{dokaz}
			
			Da dopolnimo znanje o unitarnih operatorjih, premislimo še naslednjo lemo.
			
			\begin{lema}
				\label{lem:unitarset}
				Naj bo $X$ poljuben kompleksen Hilbertov prostor ter naj bo $\mathcal{U}(X)$ množica vseh unitarnih operatorjev na $X$. Velja:
				\begin{enumerate}[a)]
					\item Če je $U\in \mathcal{U}(X)$, je tudi $U^*\in \mathcal{U}(X)$, ter $\norm{U} = \norm{U^*} = 1$.
					\item Če sta $U_1, U_2 \in \mathcal{U}(X)$, sta tudi $U_1U_2$ in $U_1^{-1}$.
					\item $\mathcal{U}(X)$ je zaprta podmnožica $B(X)$.
				\end{enumerate}
			\end{lema}
			
			\begin{dokaz}
				\begin{enumerate}[a)]
					\item Naj bo $U$ poljuben unitaren operator na $X$. Tedaj je $U^*(U^*)^* = U^*U = I$ in $(U^*)^*U^* = UU^* = I$, torej je tudi $U^*\in B(X)$ unitaren. Po izreku \ref{izr:adjlast} že vemo, da je $\norm{U} = \norm{U^*}$. Dodatno nam izrek \ref{izr:uniizo} pove, da je $U$ tudi izometrija, torej je $\norm{U} = 1$.
					\item Naj bosta $U_1$ in $U_2$ poljubna unitarna operatorja na $X$. S pomočjo izreka \ref{izr:adjlast} vidimo, da je: \begin{align*}
						U_1U_2(U_1U_2)^* &= U_1U_2U_2^*U_1^* = U_1IU_1^* = I \\
						(U_1U_2)^*U_1U_2 &= U_2^*U_1^*U_1U_2 = U_2^*IU_2 = I 
					\end{align*} torej je tudi $U_1U_2 \in \mathcal{U}(X)$.
					Ker je $U_1$ unitaren, je omejen in po izreku \ref{izr:uniizo} bijektiven, je po posledici izreka o odprti preslikavi tudi obrnljiv (znotraj $B(X)$). Dodatno, upoštevamo lemo \ref{lem:adjinv} in tako dobimo: \begin{align*}
						(U_1^{-1})^*U_1^{-1} &= (U_1^*)^{-1}U_1^{-1} = (U_1U_1^*)^{-1} = I^{-1} = I \\
						U_1^{-1}(U_1^{-1})^* &= U_1^{-1}(U_1^*)^{-1} = (U_1^*U_1)^{-1} = I^{-1} = I
					\end{align*}
					Sledi, da je $U_1^{-1}$ unitaren.
					\item Naj bo $\{U_n\}_{n\in\N}$ zaporedje s členi v $\mathcal{U}(X)$ in limito $U \in B(X)$. Potem je po točki $a)$ tega dokaza tudi zaporedje $\{U_n^*\}_{n\in\N}$ zaporedje s členi v $\mathcal{U}(X)$. Ker je adjungiranje zvezno, je $U^*\in B(X)$ limita zaporedja $\{U_n^*\}_{n\in\N}$. Naj bo $x\in X$ poljuben. Potem je $\norm{Ux}^2 = \Sp{Ux}{Ux} = \lim_{n\to\infty}\Sp{U_nx}{U_nx} = \lim_{n\to\infty}\norm{U_nx}^2 = \norm{x}^2$. Sledi, da je $U$ izometrija, po izreku \ref{izr:uniizo} pa sledi, da je potem $U^*U = I$. Na enak način kot za $U$ vidimo, da je tudi $U^*$ izometrija in posledično po izreku \ref{izr:uniizo} velja, da je $(U^*)^*U^* = UU^* = I$. Sledi, da je $U$ unitaren.
				\end{enumerate}
			\end{dokaz}
	\section{Spekter}
		Problem lastnih vrednosti matrik nam je znan že iz linearne algebre, ni pa omejen le na matrike - na enak način jih lahko definiramo tudi za operatorje na poljubnih prostorih. V tem poglavju bomo obravnavali t.~i.~ spekter operatorja nad kompleksnim Hilbertovim prostorom.
		
		\begin{definicija}
			\label{def:specter}
			Naj bo $X$ poljuben kompleksen Hilbertov prostor ter naj bo $T\in B(X)$ poljuben. \pojem{Spekter} operatorja $T$ je množica, definirana na naslednji način: $$\sigma(T) = \{\lambda\in\C~|~T-\lambda I~\text{ni obrnljiv}\}$$
		\end{definicija}
		
		V naslednji lemi preverimo, da spekter vsebuje vse lastne vrednosti operatorja.
		
		\begin{lema}
			\label{lem:eigeninspecter}
			Naj bo $X$ poljuben kompleksen Hilbertov prostor ter naj bo $T\in B(X)$ poljuben. Če je $\lambda$ lastna vrednost operatorja $T$, potem je $\lambda\in\sigma(T)$.
		\end{lema}
		
		\begin{dokaz}
			Naj bo $T\in B(X)$ poljuben ter naj bo $\lambda$ njegova lastna vrednost. Tedaj po definiciji obstaja nek neničelni vektor $x\in X$, za katerega velja: $Tx = \lambda x$. Posledično, je $x\in Ker(T-\lambda I)$, torej operator $T - \lambda I$ ni injektiven in posledično ni obrnljiv.
		\end{dokaz}
		
		\begin{opomba}
			\label{opomb:eigeninspecter}
			V končno-razsežnih vektorskih prostorih vemo, da je injektivnost linearne preslikave ekvivalentna surjektivnosti. Posledično, je spekter linearne preslikave nad končno-razsežnem kompleksnem Hilbertovim prostorom natanko množica njegovih lastnih vrednosti. V splošnem primeru to ni res - obstajajo omejeni linearni operatorji nad kompleksnimi Hilbertovimi prostori, ki ne premorejo nobene lastne vrednosti. Primer tega je desni premik $D\in B(l^2)$.
		\end{opomba}
		
		Preden določimo nekaj osnovnih lastnosti spektra operatorja, se spomnimo Liouvilleovega izreka iz kompleksne analize.
		
		\begin{izrek}
			\label{izr:Liouville}
			Naj bo $\map{f}{\C}{\C}$ poljubna cela funkcija. Če je $f$ omejena, je tudi konstantna.
		\end{izrek}
		
		\begin{izrek}
			\label{izr:specter}
			Naj bo $X$ poljuben kompleksen Hilbertov prostor. Za vsak $T\in B(X)$ velja: \begin{enumerate}[a)]
				\item $\lambda\in\sigma(T) \Rightarrow \abs{\lambda}\leq\norm{T}$.
				\item $\sigma(T)$ je kompaktna množica v $B(X)$.
				\item $\sigma(T)\neq \emptyset$.
			\end{enumerate}
		\end{izrek}
		
		\begin{dokaz}
			Tekom dokaza se bomo sklicali na izreke \ref{izr:Hilbertopinv}, \ref{izr:Hilbertinvset} in \ref{izr:Liouville}.
			\begin{enumerate}[a)]
				\item Denimo, da je $\lambda > \norm{T}$. Tedaj je $\norm{\lambda^{-1}T} < 1$ in posledično je po izreku \ref{izr:Hilbertopinv} operator $I - \lambda^{-1}T$ obrnljiv. Sledi, da je tudi operator $-\lambda(I - \lambda^{-1}T)= T - \lambda I$ obrnljiv. Sledi, da $\lambda\notin\sigma(T)$.
				\item Ker smo v točki $a)$ pokazali, da je spekter omejena množica, za dokaz te točke zadošča, da pokažemo, da je tudi zaprta. Naj bo torej $T\in B(X)$ poljuben in definirajmo preslikavo $\map{f}{\C}{B(X)}$ s predpisom $f(\lambda) = T - \lambda I$. Opazimo: Za poljubna $\mu, \lambda \in \C$ je $\norm{f(\mu) - f(\lambda)} = \abs{\mu - \lambda}\cdot\norm{I}$.
				Naj bo sedaj $\varepsilon>0$ poljuben in naj bo $\delta = \frac{\varepsilon}{\norm{I}}$. Velja: Če je $\abs{\mu - \lambda}<\delta$, je $\norm{f(\mu) - f(\lambda)} < \frac{\varepsilon}{\norm{I}}\norm{I} = \varepsilon$. Sledi, da je naša funkcija $f$ enakomerno zvezna. Opazimo še naslednje: $$\lambda\in\sigma(T) \iff T-\lambda I \in \mathcal{G}^{c}\iff f(\lambda)\in\mathcal{G}^{c} \iff \lambda\in f^{-1}(\mathcal{G}^{c})$$
				Posledično je $f^{-1}(\mathcal{G}^{c}) = \sigma(T)$. Ker je $f$ zvezna in $\mathcal{G}^{c}$ zaprta ($\mathcal{G}$ je po izreku \ref{izr:Hilbertinvset} odprta), je $f^{-1}(\mathcal{G}^{c})$ zaprta, torej je tudi $\sigma(T)$ zaprta. Po izreku Heine-Borel-Lebesgue je potem $\sigma(T)$ kompakt.
				\item Dokaz bomo izvedli s protislovjem. Naj bo $\sigma(T)=\emptyset$ za nek $T\in B(X)$. Sledi, da je za vsak $\lambda\in\C$ operator $T - \lambda I$ obrnljiv in označimo njegov inverz z $(T - \lambda I)^{-1}$. Naj bo $\map{f}{X}{\C}$ poljuben omejen linearen funkcional na $X$. Sedaj  definiramo funkcijo $\map{\varphi}{\C}{\C}$ s predpisom $\varphi(\lambda)= f((T-\lambda I)^{-1})$. V nadaljevanju bomo pokazali, da je $\varphi$ cela omejena funkcija.
				Premislimo najprej, da je cela: Naj bo $\lambda_0\in\C$ poljuben. Opazimo: \begin{align*}
					\frac{\varphi(\lambda)-\varphi(\lambda_0)}{\lambda - \lambda_0} &= \frac{1}{\lambda - \lambda_0}(f((T-\lambda I)^{-1}) - f((T-\lambda_0 I)^{-1})) \\
					&= \frac{1}{\lambda - \lambda_0}f((T-\lambda I)^{-1} - (T-\lambda_0 I)^{-1}) \\
					&= \frac{1}{\lambda - \lambda_0}f((T-\lambda_0 I)^{-1}\left((T-\lambda_0 I) - (T-\lambda I)\right)(T-\lambda I)^{-1}) \\
					&= \frac{1}{\lambda - \lambda_0}f((\lambda - \lambda_0)(T-\lambda_0 I)^{-1}(T-\lambda I)^{-1}) \\
					&= f((T-\lambda_0 I)^{-1}(T-\lambda I)^{-1})
				\end{align*}
				Posledično je \begin{align*}
					\lim_{\lambda\to \lambda_0}\frac{\varphi(\lambda)-\varphi(\lambda_0)}{\lambda - \lambda_0} &= \lim_{\lambda\to\lambda_0}f((T-\lambda_0 I)^{-1}(T-\lambda I)^{-1}) \\
					&= f(\lim_{\lambda\to \lambda_0}(T-\lambda_0 I)^{-1}(T-\lambda I)^{-1}) \\
					&= f((T-\lambda_0 I)^{-1}\lim_{\lambda\to \lambda_0}(T-\lambda I)^{-1}) \\
					&= f(((T-\lambda_0 I)^{-1})^2)
				\end{align*}
				Ker ta limita obstaja za vsak $\lambda_0\in\C$, je $\varphi$ holomorfna na $\C$, torej cela. Premislimo še, da je $\varphi$ omejena.
				Naj bo $\lambda\in\C$ in naj bo $\abs{\lambda}>\norm{T}$. Tedaj je $(T- \lambda I)^{-1} = (-\lambda)(I - \lambda^{-1}T)^{-1}$. Opazimo, da je $\norm{\lambda^{-1}T}<1$. Po izreku \ref{izr:Hilbertopinv}, je potem $(T- \lambda I)^{-1} = (-\lambda)\sum_{n = 1}^{\infty}(\lambda^{-1}T)^n$. Sledi: $$\norm{(T- \lambda I)^{-1}} \leq\frac{1}{\abs{\lambda}}\sum_{n = 1}^{\infty}\abs{\lambda}^{-n}\norm{T}^n = \frac{1}{\abs{\lambda}}\sum_{n=1}^{\infty}\left(\frac{\norm{T}}{\abs{\lambda}}\right)^n = \frac{1}{\abs{\lambda}}\frac{1}{1-\frac{\norm{T}}{\abs{\lambda}}}=\frac{1}{\abs{\lambda}-\norm{T}}$$
				V limiti, ko pošljemo $\abs{\lambda}$ proti $\infty$, gre $\frac{1}{\abs{\lambda} - \norm{T}}$ proti $0$. Posledično sklepamo, da je $\lim_{\abs{\lambda}\to\infty}\norm{(T- \lambda I)^{-1}} = 0$. Z dobljeno limito obravnavamo obnašanje $\varphi$ daleč od izhodišča: Za vsak $\lambda\in\C$ velja ocena: $\abs{\varphi(\lambda)} = \abs{f((T-\lambda I)^{-1})} \leq \norm{f}\norm{(T-\lambda I)^{-1}}$. Ko pošljemo $\abs{\lambda}$ proti $\infty$, gre izraz na desni proti $0$, torej bo enako veljalo za $\abs{\varphi(\lambda)}$ in posledično je $\lim_{\abs{\lambda}\to\infty}\varphi(\lambda)=0$. Sledi, da obstaja tak $R>0$, da je $\abs{\varphi(\lambda)}\leq 1$ za vsak $\lambda\in \overline{K_R(0)}^c$. Hkrati je $\varphi$ zvezna na $\overline{K_R(0)}$, torej je tam tudi omejena. Posledično je $\varphi$ omejena na celem $\C$.
				Po Liouvilleovem izreku \ref{izr:Liouville} je potem $\varphi$ konstantna funkcija, limita $\lim_{\abs{\lambda}\to\infty}\varphi(\lambda) = 0$ pa nam pove, da je $\varphi\equiv 0$. Sledi, da je $f((T - \lambda I)^{-1})=0;~\forall \lambda\in\C$. Ker je $f$ bil poljuben omejen linearen funkcional na $X$, sklep velja za vse omejene linearne funkcionale na $X$. Sedaj se spomnimo posledice Hahn-Banachovega izreka, ki nam pove, da obstaja tak omejen linearen funkcional $f$ nad $X$, da je $f((T- \lambda I)^{-1})=\norm{(T- \lambda I)^{-1}}=0;~\forall \lambda\in\C$ oziroma $(T- \lambda I)^{-1} = 0;~ \forall \lambda\in\C$. To nas pa privede v protislovje, saj ničelni operator ni obrnljiv.
			\end{enumerate}
		\end{dokaz}
		
		\begin{opomba}
			\label{opomb:specprimeri} Naj bo $X$ poljuben kompleksen Hilbertov prostor. Prva točka izreka \ref{izr:specter} nam pove, da za poljuben $T\in B(X)$ velja, da je $\sigma(T)\subseteq \overline{K_{\norm{T}}(0)}$. S pomočjo druge točke izreka tudi opazimo dva skrajna primera za spekter. V prvem skrajnem primeru spekter vsebuje le eno samo vrednost: $\sigma(T)=\{\lambda_0\}$, za nek $\lambda_0\in\overline{K_{\norm{T}}(0)}$. Primer takega operatorja $T$ je ravno ničelni operator: $\sigma(0) = \{0\}$. Drugi skrajni primer je, da je $\sigma(T) = \overline{K_{\norm{T}}(0)}$.
		\end{opomba}
		
		Sedaj premislimo še povezavo med spektrom poljubnega operatorja in spektrom njegovega adjungiranca.
		
		\begin{lema}
			\label{lem:adjspec}
			Naj bo $X$ poljuben kompleksen Hilbertov prostor in naj bo $T\in B(x)$ poljuben. Tedaj je $\sigma(T^*)=\{\bar{\lambda}~|~\lambda\in\sigma(T)\}$.
		\end{lema}
		
		\begin{dokaz}
			Denimo, da nek $\lambda\notin\sigma(T)$. Tedaj je operator $T-\lambda I$ obrnljiv in po lemi \ref{lem:adjinv} je potem tudi $(T-\lambda I)^* = T^* - \bar{\lambda}I$ obrnljiv. Sledi, da $\bar{\lambda}\notin\sigma(T^*)$. Na enak način premislimo, da če $\bar{\lambda}\notin\sigma(T^*)$, potem $\lambda\notin\sigma(T)$. Sledi, da je $\sigma(T^*) =\{\bar{\lambda}~|~\lambda\in\sigma(T)\}$.
		\end{dokaz}
		
		\begin{izrek}
			\label{izr:polyspec}
			Naj bo $X$ poljuben kompleksen Hilbertov prostor ter naj bo $T\in B(X)$ poljuben. Naj bo $p\in\C[X]$ poljuben nekonstanten polinom. Velja: \begin{enumerate}[a)]
				\item $\sigma(p(T))=\{p(z)~|~z\in\sigma(T)\}$.
				\item Če je $T$ obrnljiv, je $\sigma(T^{-1})=\{\lambda^{-1}~|~\lambda\in\sigma(T)\}$.
			\end{enumerate}
		\end{izrek}
		
		\begin{dokaz}
			Naj bo $T\in B(X)$ poljuben.
			\begin{enumerate}[a)]
				\item Naj bo $p\in\C[X]$ poljuben nekonstanten polinom in naj bo $\lambda\in\C$ poljuben. Definiramo polinom $q$ s predpisom $q(z) = \lambda - p(z)$. Po osnovnem izreku algebre lahko polinom $q$ zapišemo kot produkt linearnih faktorjev: $q(z) = a(z-\lambda_1)(z-\lambda_2)\ldots(z-\lambda_n)$, kjer so $\lambda_1, \lambda_2,\ldots, \lambda_n\in\C$ ničle polinoma in $a\in\C\setminus\{0\}$. Velja: \begin{align*}
					\lambda\notin \sigma(p(T)) &\iff~\text{Operator}~\lambda I - p(T)~\text{je obrnljiv} \iff q(T)~\text{je obrnljiv} \\
					&\iff a(T-\lambda_1 I)(T-\lambda_2 I)\ldots(T-\lambda_n I)~\text{je obrnljiv} \\
					&\iff (T-\lambda_i I)~\text{je obrnljiv za}~\forall i\in \{1, 2, \ldots, n\}\\
					&\iff \text{Nobena ničla polinoma}~q~\text{ni vsebovana v}~\sigma(T) \\
					&\iff q(z)\neq 0;~\forall z\in\sigma(T) \iff \lambda \neq p(z);~\forall z\in\sigma(T)
				\end{align*}
				Posledično sledi: $\sigma(p(T))=\{p(z)~|~z\in\sigma(T)\}$.
				\item Naj bo $T\in B(X)$ obrnljiv. Posledično $0\notin\sigma(T)$ in zato lahko vsak element $\sigma(T^{-1})$ zapišemo kot inverz nekega $\mu\in\C$. Opazimo, da je $\mu^{-1}I - T^{-1} = -\mu^{-1}T^{-1}(\mu I - T)$ in da je operator $\mu^{-1}T^{-1}$ obrnljiv. Sledi: \begin{align*}
					\mu^{-1}\in\sigma(T^{-1}) &\iff \mu^{-1}I - T^{-1}~\text{ni obrnljiv} \\
					&\iff -\mu^{-1}T^{-1}(\mu I - T)~\text{ni obrnljiv} \\
					&\iff (\mu I - T)~\text{ni obrnljiv} \iff \mu\in\sigma(T) \\
				\end{align*}
				Sledi, da je $\sigma(T^{-1})=\{\mu^{-1}~|~\mu\in\sigma(T)\}$.
			\end{enumerate}
		\end{dokaz}
		
		S pomočjo izreka \ref{izr:polyspec} lahko enostavno določimo spektre raznih posebnih tipov operatorjev. V naslednji lemi bomo to storili za idempotentne, nilpotentne in unitarne operatorje.
		
		\begin{lema}
			\label{lem:idenilunispec}
			Naj bo $X$ poljuben kompleksen Hilbertov prostor in naj bo $T\in B(X)$ poljuben. Velja: \begin{enumerate}[a)]
				\item Če je $T\in B(X)\setminus\{0, I\}$ idempotent (torej, če je $T^2 = T$), je $\sigma(T) = \{0, 1\}$.
				\item Če je $T\in B(X)$ nilpotent (torej, če obstaja tak $n\in \N$, da je $T^n = 0$), je $\sigma(T) = \{0\}$.
				\item Če je $T\in B(X)$ unitaren, je $\sigma(T) \subseteq \{\lambda\in\C~|~ \abs{\lambda}=1\}$.
			\end{enumerate}
		\end{lema}
		
		\begin{dokaz}
			Ključni del dokaza vseh trditev bo uporaba izreka \ref{izr:polyspec}.\begin{enumerate}[a)]
				\item Naj bo $T\in B(X)\setminus\{0, I\}$ idempotent. Ker je $T$ idempotent, zadošča enačbi $T^2 - T = 0$. Označimo polinom $p(z) = z^2 - z$ in opazimo, da potem operator $T$ zadošča enačbi $p(z) = 0$. Kot smo že omenili v opombi \ref{opomb:specprimeri}, je $\sigma(0) = \{0\}$. Po prvi točki izreka \ref{izr:polyspec} je potem $\{0\}=\sigma(0)=\sigma(p(T))=\sigma(T^2 - T) = \{\lambda^2 - \lambda~|~\lambda\in\sigma(T)\}$. Za vsak $\lambda\in\sigma(T)$ torej velja, da je $\lambda^2 - \lambda = 0$. Ta enačba ima dve rešitvi: $\lambda=0$ in $\lambda=1$. Sledi, da je $\sigma(T)\subseteq\{0, 1\}$. Hkrati po tretji točki izreka \ref{izr:specter} vemo, da $\sigma(T)\neq\emptyset$.
				Obravnavajmo naslednji možnosti: \begin{enumerate}[i)]
					\item Denimo, da $1\notin\sigma(T)$. Tedaj je operator $T- I$ obrnljiv, torej lahko enačbo $(T-I)T = T^2 - T = 0$ z leve pomnožimo z $(T-I)^{-1}$. Tako dobimo enačbo $T = 0$, kar pa nas privede v protislovje z začetno predpostavko, da je $T\in B(X\setminus\{0, I\})$. Sledi, da je $1\in\sigma(T)$
					\item Denimo sedaj, da $0\notin\sigma(T)$. Tedaj je $T$ obrnljiv, torej lahko enačbo $(T-I)T = T^2 - T = 0$ z leve pomnožimo z $T^{-1}$. Tako dobimo enačbo $T-I = 0$ oziroma $T=I$. Tudi to nas privede v protislovje z isto začetno predpostavko, kot v prejšnji točki. Sledi, da je $0\in\sigma(T)$.
				\end{enumerate}
				Sledi, da je $\sigma(T) = \{0, 1\}$.
				\item Naj bo sedaj $T\in B(X)$ nilpotent in naj bo $n\in\N$ najmanjši taki, da je $T^n = 0$. Označimo polinom $q(z)=z^n$. Tedaj operator $T$ zadošča enačbi $q(T) = 0$. Po prvi točki izreka \ref{izr:polyspec} je potem $\{0\}=\sigma(0)=\sigma(q(T))=\sigma(T^n)=\{\lambda^n~|~\lambda\in\sigma(T)\}$. Sledi, da za vsak $\lambda\in\sigma(T)$ velja, da je $\lambda^n = 0$. Ta enačba ima edino rešitev $\lambda=0$, torej je $\sigma(T)\subseteq\{0\}$. Ker po tretji točki izreka \ref{izr:specter} spekter ne more biti prazna množica, sledi, da je $\sigma(T)=0$.
				\item Naj bo sedaj $T\in \mathcal{U}(X)$. Po prvi točki leme \ref{lem:unitarset} vemo, da je $\norm{T}=1$. Posledično je $\sigma(T)\subseteq \overline{K_1(0)} = \{\lambda\in\C~|~\abs{\lambda}\leq 1\}$. Hkrati opazimo, da je tudi $\sigma(T^*) \subseteq \overline{K_1(0)}$. Sedaj upoštevamo, da je $T$ unitaren operator, torej je $U^* = U^{-1}$. Po drugi točki izreka \ref{izr:polyspec} je potem $\sigma(T) =
				\{\lambda^{-1}~|~\lambda\in\sigma(T^*)\}\subseteq \{\lambda\in\C~|~\abs{\lambda}\geq 1\}$. Ker je $\sigma(T)$ hkrati podmnožica $\overline{K_1(0)} in K_1^c(0)$, sledi, da je $\sigma(T)$ vsebovana v robu enotske krogle $K_1(0)$, torej je $\sigma(T)\subseteq\{\lambda\in\C~|~\abs{\lambda}=1\}$.
			\end{enumerate}
		\end{dokaz}
		
		Pred obravnavo spektrov normalnih in sebi-adjungiranih operatorjev vpeljimo naslednja pojma.
		
		\begin{definicija}
			\label{def:specrad}
			Naj bo $X$ poljuben kompleksen Hilbertov prostor in naj bo $T\in B(X)$ poljuben. \begin{enumerate}[a)]
				\item \pojem{Spektralni radij} operatorja $T$, označen z $r_\sigma(T)$, je definiran na naslednji način: $$r_\sigma(T) = \sup\{\abs{\lambda~|~\lambda\in\sigma(T)}\}$$
				\item \pojem{Numerični razpon} operatorja $T$, označen z $V(T)$, je definiran na naslednji način: $$V(T)=\{\Sp{Tx}{x}~|~\norm{x}=1\}$$
			\end{enumerate}
		\end{definicija}
		
		V primeru normalnih operatorjev velja naslednja zveza med spektrom in numeričnim razponom.
		
		\begin{lema}
			\label{lem:normalnumrazspec}
			Naj bo $(X, \Sp{.}{.})$ poljuben kompleksen Hilbertov prostor in naj bo $T\in B(X)$ poljuben. Če je $T$ normalen, je $\sigma(T)\subseteq \overline{V(T)}$.
		\end{lema}
		
		\begin{dokaz}
			Naj bo $\lambda\in\sigma(T)$ poljuben. Preverimo, da je operator $T-\lambda I$ tudi normalen operator: \begin{align*}
				(T-\lambda I)(T - \lambda I)^* &= (T-\lambda I)(T^*-\bar{\lambda} I) \\
				&= TT^* - \bar{\lambda}T -\lambda T^* + \lambda\bar{\lambda} I \\
				&= T^*T -\lambda T^* -\bar{\lambda}T + \bar{\lambda}\lambda I \\
				(T-\lambda I)^*(T - \lambda I) &= (T^*-\bar{\lambda} I)(T-\lambda I) \\
				&= T^*T -\lambda T^* -\bar{\lambda}T + \bar{\lambda}\lambda I \\
			\end{align*}
			Sledi, da je $(T-\lambda I)(T - \lambda I)^* = (T-\lambda I)^*(T - \lambda I)$, torej je $(T-\lambda I)$ normalen. Ker operator $(T - \lambda I)$ ni obrnljiv, po posledici \ref{posl:normalinv} za vsak $r>0$ obstaja nek $x_r$, taki, da je $\norm{(T-\lambda I)x_r}< r\norm{x_r}$. Posledično v $X$ obstaja zaporedje $\{x_n\}_{n\in\N};~\norm{x_n} = 1;~\forall n\in\N$, za katero je $\lim_{n\to\infty}\norm{(T-\lambda I)x_n} = 0$.
			Po neenakosti Cauchy-Schwarz-Bunyakowsky je $\forall n\in\N \abs{\Sp{(T-\lambda I)x_n}{x_n}} \leq \norm{(T-\lambda I)x_n}\cdot\norm{x_n} = \norm{(T-\lambda I)x_n}$. Posledično sledi, da je $\lim_{n\to\infty}\Sp{(T-\lambda I)x_n}{x_n} = 0$. Slednjo enakost razpišemo v $\lim_{n\to\infty}(\Sp{Tx_n}{x_n}-\lambda\Sp{x_n}{x_n})=0$ in upoštevamo, da je $\Sp{x_n}{x_n} = 1;~\forall n\in\N$. Tako dobimo, da je $\lim_{n\to\infty}\Sp{Tx_n}{x_n} = \lambda$. Sledi, da je $\lambda \in \overline{V(T)}$.
		\end{dokaz}
		
		Naslednji izrek opiše lastnosti spektrov in numeričnih razponov sebi-adjungiranih operatorjev.
		
		\begin{izrek}
			\label{izr:sebiadjnumraz}
			Naj bo $(X, \Sp{.}{.})$ poljuben kompleksen Hilbertov prostor in naj bo $T\in \mathcal{S}(X)$ poljuben. Velja: \begin{enumerate}[a)]
				\item $V(T)\subseteq \R$.
				\item $\sigma(T)\subseteq \R$.
				\item $\norm{T}\in\sigma(T)$ ali $-\norm{T}\in\sigma(T)$.
				\item $r_\sigma(T)=\sup\{\abs{t}~|~t\in V(T)\} = \norm{T}$.
				\item $\inf\{\lambda~|~\lambda\in\sigma(T)\} \leq t \leq \sup\{\lambda~|~\lambda\in\sigma(T)\};~\forall t\in V(T)$.
			\end{enumerate}
		\end{izrek}
		
		\begin{dokaz}
			Naj bo $T\in \mathcal{S}(X)$ poljuben.
			\begin{enumerate}[a)]
				\item 
			\end{enumerate}
		\end{dokaz}
		
	\begin{thebibliography}{99}
		\bibitem{bib:Rynne} B.~P.~Rynne,~M.~A.~Youngson, \emph{Linear functional analysis},~Springer-Verlag London,~Ltd.,~London,~2008.
	\end{thebibliography}
	
\end{document}